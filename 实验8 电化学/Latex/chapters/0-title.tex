\begin{titlepage}
% 页眉
\thispagestyle{plain}
% 校徽图片
\begin{figure}[h]
    \centering
    \includegraphics{pku.png}
\end{figure}
\vspace{20pt}
% 标题
\centerline{\zihao{-0} \textsf{物理化学实验报告}}
\vspace{20pt} % 空行
\begin{center}
    \begin{tabular}{cc}
        % 题目
        
        \addcell[2]{题目:\ } & \addcell[2]{循环伏安法解析电极极化} \\
        \cline{2-2}\\
        
    \end{tabular}
\end{center}
\vspace{20pt} % 空行
\begin{center}
    \doublespacing
    \begin{tabular}{cp{5cm}}
        % 姓名
        \addcell{姓\phantom{空格}名:\ } & \addcell{王子宸} \\
        \cline{2-2}
        % 学号
        \addcell{学\phantom{空格}号:\ } & \addcell{2100011873}\\
        \cline{2-2}
        % 组别
        \addcell{组\phantom{空格}别:\ } & \addcell{周四19组8号} \\
        \cline{2-2}
        % 实验日期
        \multirow{2}{*}{\addcell{实验日期:\ }} & \addcell{\zhdate{2023/11/23}}\\
        \cline{2-2}
        % 室温
        \addcell{室\phantom{空格}温:\ } & \addcell{20.8\si{{}^\circ C}}\\
        \cline{2-2}
        % 大气压强
        \addcell{大气压强:\ } & \addcell{101.2\si{kPa}}\\
        \cline{2-2}
    \end{tabular}
    \begin{tabular*}{\textwidth}{c}
        \\
        \\
        \hline % 分割线
    \end{tabular*}
\end{center}
% 摘要
\textsf{摘\ \ 要}\ \ 本实验通过循环伏安法(CV)研究了铂电极的电化学行为。首先,使用 $0.05\mathrm{M}$ 硫酸溶液在氮气饱和条件下对铂电极进行活化,通过CV曲线计算得到扫速为分别为 $0.1,0.2,0.5\mathrm{~V\cdot s^{-1}}$ 时,铂电极表面单层氢原子吸附的电量分别为 $4.85,4.23,3.79 \times 10^{-6}\ \mathrm{C}$,从而推算出铂电极的电化学活性面积为 $2.31, 2.02, 1.81\ \mathrm{mm}^2$。在不同搅拌速率和保护气氛(氮气和氧气)下测量CV曲线,发现扫描速度的增大导致电流峰值增大,而电压峰值几乎保持不变;在氧气饱和下,随着搅拌速率的增大,低电势区的电流明显增大。此外,分别测定了氧气还原反应(ORR)和甲醇氧化反应(MOR)的CV曲线,求得直接甲醇燃料电池(DMFC)的最大输出功率为 $5.034 \times 10^{-7}\mathrm{~W}$,在工作电压 $0.282\mathrm{~V}$ 处达到。本实验为电极材料的性能评估和燃料电池的优化提供了重要数据。

% 关键字
\noindent\textsf{关键词}\ \ 物理化学实验; 循环伏安法;直接甲醇燃料电池;电化学活性面积;电极极化
\end{titlepage}