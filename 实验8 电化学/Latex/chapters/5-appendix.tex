\section{附录:数据处理代码}

本人编写了一套完整的数据处理代码,在Jupyter Notebook中打开,可以做到指定文件名后一键生成所有图像与数据计算结果。

\subsection{数据的预处理与函数的设置}

\subsubsection{数据的预处理}\label{cap:1-1}

\begin{minted}[frame=lines, framesep=2mm, baselinestretch=1.2, fontsize=\footnotesize, linenos]{python}
import numpy as np
import pandas as pd
import matplotlib.pyplot as plt
import matplotlib.ticker as ticker
from matplotlib.ticker import FuncFormatter
from abc import ABC, abstractmethod
import re

class AbstractSegment(ABC):
    def __init__(self, info_lines, data_lines, number):
        self.number = number
        self.info = self.get_info(info_lines)
        self.data = self.get_data(data_lines)
    
    @abstractmethod
    def get_data(self, lines):
        pass
    
    @abstractmethod
    def get_info(self, lines):
        pass

class Segment(AbstractSegment):
    def __init__(self, info_lines, data_lines, number):
        super().__init__(info_lines, data_lines, number)

    def get_data(self, lines):
        data = [line.strip().split(', ') for line in lines[1:] if line.strip()]
        return pd.DataFrame(data, columns=['Potential/V', 'Current/A', 'Charge/C'])\
                .astype({'Potential/V': float, 'Current/A': float, 'Charge/C': float})

    def get_info(self, lines):
        Eps, ips, Ahs = [], [], []
        for line in lines:
            if line.startswith('Ep'):
                Eps.append(line.split('=')[1].strip())
            elif line.startswith('ip'):
                ips.append(line.split('=')[1].strip())
            elif line.startswith('Ah'):
                Ahs.append(line.split('=')[1].strip())

        segment_info = {
            'Segment': self.number,
            'peaks': [{'Ep':e, 'ip':i, 'Ah':a} for e, i, a in zip(Eps, ips, Ahs)] \
                        if len(Eps)*len(ips)*len(Ahs) != 0 else None,
        }

        return segment_info

class AbstractCyclicVoltammetry(ABC):
    def __init__(self, file_name):
        self.name = file_name.split('.')[0]
        self.text = self.read_file(file_name)
        self.inifo = self.initialize()
        self.segifo = self.init_segments()

    @abstractmethod
    def read_file(self, file_name):
        pass

    @abstractmethod
    def initialize(self):
        pass

    @abstractmethod
    def init_segments(self):
        pass
    
    @abstractmethod
    def get_all_data(self):
        pass

    @abstractmethod
    def get_segment_data(self, segment_number):
        pass

    @abstractmethod
    def get_all_info(self):
        pass

    @abstractmethod
    def get_segment_info(self, segment_number):
        pass
        
    @abstractmethod
    def get_initial_info(self):
        pass


class CyclicVoltammetry(AbstractCyclicVoltammetry):
    def __init__(self, file_name):
        super().__init__(file_name)

    def read_file(self, file_name):
        with open(file_name) as f:
            lines = f.readlines()
        return lines
    
    def initialize(self):
        initial_info = {
            'Name': self.name,
            'Init E (V)': None,
            'High E (V)': None,
            'Low E (V)': None,
            'Scan Rate (V/s)': None,
            'Segments': None,
            'Sample Interval (V)': None,
            'Sensitivity (A/V)': None,
        }

        combined_text = "\n".join(self.text).split('Segment 1:')[0]

        init_e_pattern = r"Init E \(V\) = ([\-\d\.]+)"
        high_e_pattern = r"High E \(V\) = ([\-\d\.]+)"
        low_e_pattern = r"Low E \(V\) = ([\-\d\.]+)"
        scan_rate_pattern = r"Scan Rate \(V/s\) = ([\-\d\.e]+)"
        segments_pattern = r"Segment = (\d+)"
        sample_interval_pattern = r"Sample Interval \(V\) = ([\-\d\.e]+)"
        sensitivity_pattern = r"Sensitivity \(A/V\) = ([\-\d\.e\-]+)"

        initial_info['Init E (V)'] = re.search(init_e_pattern, combined_text).group(1)
        initial_info['High E (V)'] = re.search(high_e_pattern, combined_text).group(1)
        initial_info['Low E (V)'] = re.search(low_e_pattern, combined_text).group(1)
        initial_info['Scan Rate (V/s)'] = re.search(scan_rate_pattern, combined_text).group(1)
        initial_info['Segments'] = int(re.search(segments_pattern, combined_text).group(1))
        initial_info['Sample Interval (V)'] = re.search(sample_interval_pattern, combined_text).group(1)
        initial_info['Sensitivity (A/V)'] = re.search(sensitivity_pattern, combined_text).group(1)

        return initial_info

    def init_segments(self):
        segments = []
        
        for i, line in enumerate(self.text):
            if line.startswith('Segment 1:'):
                segments_text = self.text[i:]
                break
        else:
            raise ValueError('Find segment text failed')
        
        for i, line in enumerate(segments_text):
            if line.startswith('Potential/V, Current/A, Charge/C'):
                segments_info_text = segments_text[:i]
                segments_data_text = segments_text[i+1:]
                break
        
        segments_info, segments_data = [], []
        segment_info_index, segment_data_index = [], []

        for i, line in enumerate(segments_info_text):
            if line.startswith('Segment '):
                segment_info_index.append(i)

        for i, line in enumerate(segments_data_text):
            if line.startswith('Segment '):
                segment_data_index.append(i)

        for i in range(len(segment_info_index)):
            if i == len(segment_info_index) - 1:
                segments_info.append(segments_info_text[segment_info_index[i]:])
            else:
                segments_info.append(segments_info_text[segment_info_index[i]:segment_info_index[i+1]])

        for i in range(len(segment_data_index)):
            if i == len(segment_data_index) - 1:
                segments_data.append(segments_data_text[segment_data_index[i]:])
            else:
                segments_data.append(segments_data_text[segment_data_index[i]:segment_data_index[i+1]])

        for i, j in zip(segments_info, segments_data):
            segments.append(Segment(i, j, len(segments) + 1))

        return segments
    
    
    def get_all_data(self):
        all_data = [segment.data for segment in self.segifo]
        return all_data

    def get_segment_data(self, segment_number):
        if segment_number <= len(self.segifo):
            return self.segifo[segment_number - 1].data
        else:
            raise ValueError("Segment number out of range")

    def get_all_info(self):
        all_info = [segment.info for segment in self.segifo]
        return all_info

    def get_segment_info(self, segment_number):
        if segment_number <= len(self.segifo):
            return self.segifo[segment_number - 1].info
        else:
            raise ValueError("Segment number out of range")
        
    def get_initial_info(self):
        return self.inifo

file_names = ['blank1.txt', 
              'blank2.txt',
              'blank5.txt',
              'MeOH.txt',
              'O2-no.txt',
              'O2-mid.txt',
              'O2-fast.txt']

files = {name: CyclicVoltammetry(name) for name in file_names}
files['blank1.txt'].get_all_data()[0]
\end{minted}

\subsubsection{函数的设置}\label{cap:1-2}
\begin{minted}[frame=lines, framesep=2mm, baselinestretch=1.2, fontsize=\footnotesize, linenos]{python}

def merge_df(ORR, df2):
    df2 = pd.concat([ORR.iloc[[-1]], df2], ignore_index=True)
    df1 = pd.concat([df2.iloc[[-1]], ORR], ignore_index=True)
    return df1, df2

def float_to_sci(f: float, digits=2):
    if f == 0:
        return '0.' + '0' * digits
    m, e = '%e' % f, int(np.floor(np.log10(abs(f))))
    # return r'...'
    if 0 <= abs(e) < 3:
        return f'{f:.{digits}f}'
    if f < 0:
        digits += 1
    return r'${}\times 10^{{{}}}$'.format(m[:digits + 2], e)

def find_cross_point(df1, df2, n_poly=2):
    from numpy.polynomial import Polynomial
    # 假设交点附近是线性关系,我们可以使用线性拟合(或多项式拟合)
    # 找到两条曲线的重叠区域
    max_min = max(min(df1['Potential/V']), min(df2['Potential/V']))
    min_max = min(max(df1['Potential/V']), max(df2['Potential/V']))

    # 筛选交点附近的数据点
    df1_near = df1[(df1['Potential/V'] >= max_min) & (df1['Potential/V'] <= min_max)]
    df2_near = df2[(df2['Potential/V'] >= max_min) & (df2['Potential/V'] <= min_max)]

    # 进行线性拟合
    poly1 = Polynomial.fit(df1_near['Potential/V'], df1_near['Current/A'], n_poly) 
    poly2 = Polynomial.fit(df2_near['Potential/V'], df2_near['Current/A'], n_poly)

    # 解方程求交点
    roots = (poly1 - poly2).roots()

    # 找到合适的根(在两条曲线的重叠区域内)
    for r in roots:
        if max_min <= r <= min_max:
            x_intercept = r
            y_intercept = poly1(x_intercept)
            break

    return x_intercept, y_intercept

def save_data(dfs, name):
    df = pd.concat(dfs, ignore_index=True, axis=1)
    df.to_csv(name + '.csv', index=False, header=False)

\end{minted}

\subsection{氮气饱和下测定不同扫速的 CV 曲线}\label{cap:2}

\begin{minted}[frame=lines, framesep=2mm, baselinestretch=1.2, fontsize=\footnotesize, linenos]{python}
dfs = []
for name in file_names[:3]:
    df = files[name].get_all_data()[-2:]
    df[0], df[1] = merge_df(df[0].iloc[:,[0, 1]], df[1].iloc[:,[0, 1]])
    dfs += df

def format_y(value, tick_number):
    return float_to_sci(value)

def format_x(value, tick_number):
    return f'{value:.3f}'

colors = ['red', 'red', 'blue', 'blue', 'green', 'green', 'black', 'black']

plt.rcParams['font.family'] = 'Times New Roman'
plt.rcParams['mathtext.fontset'] = 'custom'
plt.rcParams['mathtext.rm'] = 'Times New Roman'
plt.rcParams['mathtext.it'] = 'Times New Roman:italic'
plt.rcParams['mathtext.bf'] = 'Times New Roman:bold'
plt.rcParams['font.size'] = 16
plt.rcParams['axes.labelsize'] = 18
plt.rcParams['xtick.labelsize'] = 16
plt.rcParams['ytick.labelsize'] = 16

plt.figure(figsize=(8, 6), dpi=1200)

j = 0
for i in range(len(dfs)):
    if i % 2 == 0:
        label = r'$v=$' + '0.{}0 V/s'.format(file_names[j].split('.')[0][-1])
        j += 1
    else:
        label = None
    plt.plot(dfs[i]['Potential/V'], -dfs[i]['Current/A']*1e6, 
             label=label, color=colors[i], linewidth=1.5)

plt.xlabel(r'Potential / V vs SCE')
plt.ylabel(r'Current / $10^{-6}$A')
plt.legend(frameon=False)

plt.gca().xaxis.set_major_formatter(FuncFormatter(format_x))
plt.gca().yaxis.set_major_formatter(FuncFormatter(format_y))
plt.gca().xaxis.set_minor_locator(ticker.AutoMinorLocator())
plt.gca().yaxis.set_minor_locator(ticker.AutoMinorLocator())
plt.grid(which='major', alpha=0.5)
plt.grid(which='minor', alpha=0.2)

plt.show()    
\end{minted}

\subsection{不同条件下铂电极表面的氧还原反应的 CV 曲线}

\begin{minted}[frame=lines, framesep=2mm, baselinestretch=1.2, fontsize=\footnotesize, linenos]{python}
df4s = []
filessss = [file_names[0]] + file_names[-3:]
for name in filessss:

    df = files[name].get_all_data()[-2:]
    df4s += [df[1][df[1]['Potential/V'] > -0.26]]

def format_y(value, tick_number):
    return float_to_sci(value)

def format_x(value, tick_number):
    return f'{value:.3f}'

labels = [
        r'N${}_2$ + $\neg$ stirring', r'$\neg$ O${}_2$ + $\neg$ stirring', r'O${}_2$ + $\neg$ stirring', r'O${}_2$ + stirring'
        ]

colors = ['blue', 'red', 'green', 'black', 'orange', 'purple']

# 设置 matplotlib 参数
plt.rcParams['font.family'] = 'Times New Roman'
plt.rcParams['mathtext.fontset'] = 'custom'
plt.rcParams['mathtext.rm'] = 'Times New Roman'
plt.rcParams['mathtext.it'] = 'Times New Roman:italic'
plt.rcParams['mathtext.bf'] = 'Times New Roman:bold'
plt.rcParams['font.size'] = 16
plt.rcParams['axes.labelsize'] = 18
plt.rcParams['xtick.labelsize'] = 16
plt.rcParams['ytick.labelsize'] = 16

# 绘图
plt.figure(figsize=(8, 6), dpi=1200)

j = 0
for i in range(len(df4s)):

    plt.plot(df4s[i]['Potential/V'], -df4s[i]['Current/A']*1e5, 
             label=labels[i], color=colors[i], linewidth=1.5)

# 设置标签和图例
plt.xlabel(r'Potential / V vs SCE')
plt.ylabel(r'Current / $10^{-5}$A')
plt.legend(frameon=False)

# 设置网格线
plt.gca().xaxis.set_major_formatter(FuncFormatter(format_x))
plt.gca().yaxis.set_major_formatter(FuncFormatter(format_y))
plt.gca().xaxis.set_minor_locator(ticker.AutoMinorLocator())
plt.gca().yaxis.set_minor_locator(ticker.AutoMinorLocator())
plt.grid(which='major', alpha=0.5)
plt.grid(which='minor', alpha=0.2)

# 显示图形
plt.show()
\end{minted}

\begin{minted}[frame=lines, framesep=2mm, baselinestretch=1.2, fontsize=\footnotesize, linenos]{python}
df3s = []
for name in [file_names[0], 'MeOH.txt']:
    if name == file_names[1]:
        df = files[name].get_all_data()[-2:]
        # 删除Potential/V列的小于-0.25的行
        df[0], df[1] = merge_df(df[0].iloc[:,[0, 1]], df[1].iloc[:,[0, 1]])
        df3s += df
    else:
        df = files[name].get_all_data()[-2:]
        df[0], df[1] = merge_df(df[0].iloc[:,[0, 1]], df[1].iloc[:,[0, 1]])
        df3s += df

O2 = df3s[1].iloc[:,[0, 1]].copy()
N2 = df3s[3].iloc[:,[0, 1]].copy()

O2 = O2[(O2['Potential/V'] >= 0) & (O2['Potential/V'] <= 0.1)]
N2 = N2[(N2['Potential/V'] >= 0) & (N2['Potential/V'] <= 0.1)]

x_intercept, y_intercept = find_cross_point(O2, N2)
print(x_intercept, y_intercept)


def format_y(value, tick_number):
    return float_to_sci(value)

def format_x(value, tick_number):
    return f'{value:.3f}'

labels = [r'N${}_2$', r'O${}_2$']
colors = ['blue', 'blue', 'red', 'red', 'green', 'green', 'black', 'black']

# 设置 matplotlib 参数
plt.rcParams['font.family'] = 'Times New Roman'
plt.rcParams['mathtext.fontset'] = 'custom'
plt.rcParams['mathtext.rm'] = 'Times New Roman'
plt.rcParams['mathtext.it'] = 'Times New Roman:italic'
plt.rcParams['mathtext.bf'] = 'Times New Roman:bold'
plt.rcParams['font.size'] = 16
plt.rcParams['axes.labelsize'] = 18
plt.rcParams['xtick.labelsize'] = 16
plt.rcParams['ytick.labelsize'] = 16

# 绘图
plt.figure(figsize=(8, 6), dpi=1200)

j = 0
for i in range(len(df3s)):
    if i % 2 == 0:
        label = labels[j]
        j += 1
    else:
        label = None

    plt.plot(df3s[i]['Potential/V'], -df3s[i]['Current/A']*1e6, 
             label=label, color=colors[i], linewidth=1.5)
plt.plot(x_intercept, -y_intercept*1e6, 'o', color='black', markersize=6, label=r'$E_{ox}$')
# 设置标签和图例
plt.xlabel(r'Potential / V vs SCE')
plt.ylabel(r'Current / $10^{-6}$A')
plt.legend(frameon=False)
# label为起始氧化电位


# 设置网格线
plt.gca().xaxis.set_major_formatter(FuncFormatter(format_x))
plt.gca().yaxis.set_major_formatter(FuncFormatter(format_y))
plt.gca().xaxis.set_minor_locator(ticker.AutoMinorLocator())
plt.gca().yaxis.set_minor_locator(ticker.AutoMinorLocator())
plt.grid(which='major', alpha=0.5)
plt.grid(which='minor', alpha=0.2)

# 显示图形
plt.show()
\end{minted}

\subsection{铂电极表面的甲醇电化学氧化反应}

\begin{minted}[frame=lines, framesep=2mm, baselinestretch=1.2, fontsize=\footnotesize, linenos]{python}
N2_cathode = df2s[0].iloc[:,[0, 1]].copy()
N2_anode = df2s[1].iloc[:,[0, 1]].copy()
O2_cathode = df2s[2].iloc[:,[0, 1]].copy()
O2_anode = df2s[3].iloc[:,[0, 1]].copy()

N2_cathode = N2_cathode[(N2_cathode['Potential/V'] >= 0.5) & (N2_cathode['Potential/V'] <= 0.8)]
N2_anode = N2_anode[(N2_anode['Potential/V'] >= 0.5) & (N2_anode['Potential/V'] <= 0.8)]
O2_cathode = O2_cathode[(O2_cathode['Potential/V'] >= 0.5) & (O2_cathode['Potential/V'] <= 0.8)]
O2_anode = O2_anode[(O2_anode['Potential/V'] >= 0.5) & (O2_anode['Potential/V'] <= 0.8)]

E_red_1 = find_cross_point(N2_cathode, O2_cathode, n_poly=4)
E_red_2 = find_cross_point(N2_anode, O2_anode, n_poly=4)

print(E_red_1)
print(E_red_2)

def format_y(value, tick_number):
    return float_to_sci(value)

def format_x(value, tick_number):
    return f'{value:.3f}'

labels = [r'N${}_2$', r'O${}_2$']
colors = ['blue', 'blue', 'red', 'red', 'green', 'green', 'black', 'black']

# 设置 matplotlib 参数
plt.rcParams['font.family'] = 'Times New Roman'
plt.rcParams['mathtext.fontset'] = 'custom'
plt.rcParams['mathtext.rm'] = 'Times New Roman'
plt.rcParams['mathtext.it'] = 'Times New Roman:italic'
plt.rcParams['mathtext.bf'] = 'Times New Roman:bold'
plt.rcParams['font.size'] = 16
plt.rcParams['axes.labelsize'] = 18
plt.rcParams['xtick.labelsize'] = 16
plt.rcParams['ytick.labelsize'] = 16

# 绘图
plt.figure(figsize=(8, 6), dpi=1200)

j = 0
for i in range(len(df2s)):
    if i % 2 == 0:
        label = labels[j]
        j += 1
    else:
        label = None

    plt.plot(df2s[i]['Potential/V'], -df2s[i]['Current/A']*1e6, 
             label=label, color=colors[i], linewidth=1.5)

plt.plot(E_red_2[0], -E_red_2[1]*1e6, 'x', color='black', markersize=6, label=r'$E_{red,1}$')
plt.plot(E_red_1[0], -E_red_1[1]*1e6, 'o', color='black', markersize=6, label=r'$E_{red,2}$')

# 设置标签和图例
plt.xlabel(r'Potential / V vs SCE')
plt.ylabel(r'Current / $10^{-6}$A')
plt.legend(frameon=False)

# 设置网格线
plt.gca().xaxis.set_major_formatter(FuncFormatter(format_x))
plt.gca().yaxis.set_major_formatter(FuncFormatter(format_y))
plt.gca().xaxis.set_minor_locator(ticker.AutoMinorLocator())
plt.gca().yaxis.set_minor_locator(ticker.AutoMinorLocator())
plt.grid(which='major', alpha=0.5)
plt.grid(which='minor', alpha=0.2)

# 显示图形
plt.show()
\end{minted}

\subsection{铂电极的电化学活性面积}

\begin{minted}[frame=lines, framesep=2mm, baselinestretch=1.2, fontsize=\footnotesize, linenos]{python}

def draw_cal_ESC(df, n_points, mode, v):
    from scipy.interpolate import interp1d
    from scipy.integrate import trapz

    def format_y(value, tick_number):
        return float_to_sci(value)

    def format_x(value, tick_number):
        return f'{value:.3f}'

    colors = ['red', 'red', 'blue', 'blue', 'green', 'green', 'black', 'black']
    label = str(v) + ' V/s'

    # 设置 matplotlib 参数
    plt.rcParams['font.family'] = 'Times New Roman'
    plt.rcParams['mathtext.fontset'] = 'custom'
    plt.rcParams['mathtext.rm'] = 'Times New Roman'
    plt.rcParams['mathtext.it'] = 'Times New Roman:italic'
    plt.rcParams['mathtext.bf'] = 'Times New Roman:bold'
    plt.rcParams['font.size'] = 16
    plt.rcParams['axes.labelsize'] = 18
    plt.rcParams['xtick.labelsize'] = 16
    plt.rcParams['ytick.labelsize'] = 16

    # 绘图
    plt.figure(figsize=(8, 6), dpi=1200)

    plt.plot(df['Potential/V'], -df['Current/A']*1e6, 
                label=label, color=colors[i], linewidth=1.5)

    if mode == 'c':
        dff = df[(df['Potential/V'] < 0.4) & (df['Potential/V'] > -0.1)]
        x_ = dff.nlargest(n_points, 'Current/A')['Potential/V'].mean()
        print(x_)
        y_constant = dff.nlargest(n_points, 'Current/A')['Current/A'].mean()
    elif mode == 'a':
        dff = df[(df['Potential/V'] < 0.4) & (df['Potential/V'] > -0.1)]
        x_ = dff.nsmallest(n_points, 'Current/A')['Potential/V'].mean()
        y_constant = dff.nsmallest(n_points, 'Current/A')['Current/A'].mean()

    x_vals = df[df['Potential/V'] < 0.4]['Potential/V']
    y_vals = df[df['Potential/V'] < 0.4]['Current/A'] * -1
    x_vals_masked = np.ma.masked_where(x_vals >= x_, x_vals)
    y_vals_masked = np.ma.masked_where(x_vals >= x_, y_vals)

    area = trapz(y_vals_masked, x_vals_masked) 

    y_label = r'Current = {} A'.format(float_to_sci(y_constant))
    plt.axhline(y=-y_constant * 1e6, color='r', linestyle='--', label=y_label)
    y_constant = -y_constant * 1e6

    # 求另一个交点
    x_vals_2 = df[df['Potential/V'] < 0]['Potential/V']
    y_vals_2 = df[df['Potential/V'] < 0]['Current/A'] * -1
    if mode == 'a':
        x_vals_masked_2 = np.ma.masked_where(y_vals_2*1e6 <= y_constant, x_vals_2)
        y_vals_masked_2 = np.ma.masked_where(y_vals_2*1e6 <= y_constant, y_vals_2)
    elif mode == 'c':
        x_vals_masked_2 = np.ma.masked_where(y_vals_2*1e6 >= y_constant, x_vals_2)
        y_vals_masked_2 = np.ma.masked_where(y_vals_2*1e6 >= y_constant, y_vals_2)
    
    # 选取x_vals_masked_2中的数字元素
    x_vals_masked_2 = x_vals_masked_2[~x_vals_masked_2.mask]
    y_vals_masked_2 = y_vals_masked_2[~y_vals_masked_2.mask]

    #从x_vals_masked中去除x_vals_masked_2中的元素
    x_vals_masked = x_vals_masked[~np.isin(x_vals_masked, x_vals_masked_2)]
    y_vals_masked = y_vals_masked[~np.isin(y_vals_masked, y_vals_masked_2)]

    plt.fill_between(x_vals_masked, y_vals_masked * 1e6, y_constant, color='grey', alpha=0.5)

    plt.xlabel(r'Potential / V vs SCE')
    plt.ylabel(r'Current / $10^{-6}$A')
    plt.legend(frameon=False)
    
    plt.gca().xaxis.set_major_formatter(FuncFormatter(format_x))
    plt.gca().yaxis.set_major_formatter(FuncFormatter(format_y))
    plt.gca().xaxis.set_minor_locator(ticker.AutoMinorLocator())
    plt.gca().yaxis.set_minor_locator(ticker.AutoMinorLocator())
    plt.grid(which='major', alpha=0.5)
    plt.grid(which='minor', alpha=0.2)

    plt.show()
    
    q = area / v
    esa = q / 2.1e-6
    print('Area:', float_to_sci(area))
    print('Q:', float_to_sci(q))
    print('ESA:', float_to_sci(esa))

def draw_ESC(files, name):
    print(name)
    df1, df2 = files[name].get_all_data()[-2:]
    df1, df2 = merge_df(df1.iloc[:,[0, 1]], df2.iloc[:,[0, 1]])
    df1 = pd.concat([df2, df1.iloc[[1]]], ignore_index=True)
    df1['Current/A'] = df1['Current/A'] * -1
    v = float(files[name].get_initial_info()['Scan Rate (V/s)'])
    draw_cal_ESC(df2, 1, 'c', v)
    draw_cal_ESC(df1, 1, 'a', v)

for name in file_names[:3]:
    draw_ESC(files, name)
\end{minted}

\subsection{直接甲醇燃料电池输出电压-输出功率曲线}

\begin{minted}[frame=lines, framesep=2mm, baselinestretch=1.2, fontsize=\footnotesize, linenos]{python}
from numpy.polynomial.polynomial import Polynomial
from scipy.interpolate import interp1d

ep_ORR = files['MeOH.txt'].get_all_data()[-1].iloc[:,[0, 1]].copy()
ep_MOR = files['O2-no.txt'].get_all_data()[-2].iloc[:,[0, 1]].copy()
background = [_.iloc[:,[0, 1]].copy() for _ in files['blank1.txt'].get_all_data()[-2:]]
# 扣除背景
ep_ORR['Current/A'] = (ep_ORR['Current/A'] - background[1]['Current/A']) * -1
ep_MOR['Current/A'] = ep_MOR['Current/A'] - background[0]['Current/A']

def interpolate_potential(df, current_value):
    if not (df['Current/A'].min() <= current_value <= df['Current/A'].max()):
        raise ValueError("Current value is out of the DataFrame's range.")

    f = interp1d(df['Current/A'], df['Potential/V'], kind='linear')

    potential_value = f(current_value)
    return potential_value

def draw_ep(ORR, MOR):
    def format_y(value, tick_number):
        return float_to_sci(value)

    def format_x(value, tick_number):
        return f'{value:.3f}'

    colors = ['red', 'red', 'blue', 'blue', 'green', 'green', 'black', 'black']

    plt.rcParams['font.family'] = 'Times New Roman'
    plt.rcParams['mathtext.fontset'] = 'custom'
    plt.rcParams['mathtext.rm'] = 'Times New Roman'
    plt.rcParams['mathtext.it'] = 'Times New Roman:italic'
    plt.rcParams['mathtext.bf'] = 'Times New Roman:bold'
    plt.rcParams['font.size'] = 16
    plt.rcParams['axes.labelsize'] = 18
    plt.rcParams['xtick.labelsize'] = 16
    plt.rcParams['ytick.labelsize'] = 16

    plt.figure(figsize=(8, 6), dpi=1200)

    plt.plot(ORR['Potential/V'], ORR['Current/A']*1e6, 
                label='ORR', color=colors[3], linewidth=1.5)
    plt.plot(MOR['Potential/V'], MOR['Current/A']*1e6,
                label='MOR', color=colors[5], linewidth=1.5)
    
    ORR_ = ORR[(ORR['Potential/V'] < 0.44) & (ORR['Potential/V'] > 0.41)].copy().reset_index(drop=True)
    MOR_ = MOR[(MOR['Potential/V'] < 0.44) & (MOR['Potential/V'] > 0.41)].copy().iloc[::-1].reset_index(drop=True)

    orr_peak_potential = ORR.nlargest(1, 'Current/A')['Potential/V'].values[0]
    mor_peak_potential = MOR.nlargest(1, 'Current/A')['Potential/V'].values[0]

    oor_zeroc_potential = ORR[ORR['Current/A'] > 0].nsmallest(1, 'Current/A')['Potential/V'].values[0]
    mor_zeroc_potential = MOR[MOR['Current/A'] > 0].nsmallest(1, 'Current/A')['Potential/V'].values[0]
    
    print(orr_peak_potential, mor_peak_potential)

    x_intercept, y_intercept = find_cross_point(ORR_, MOR_)
    print(x_intercept)
    print(y_intercept)

    cross_point_df = pd.DataFrame({'Potential/V': [x_intercept], 'Current/A': [y_intercept]})

    # 合并数据并排序
    ORR = pd.concat([ORR, cross_point_df]).sort_values(by='Potential/V').reset_index(drop=True)
    MOR = pd.concat([MOR, cross_point_df]).sort_values(by='Potential/V').reset_index(drop=True)
    
    ORR__ = ORR[(ORR['Potential/V'] < orr_peak_potential) & (ORR['Current/A'] <= y_intercept) & (ORR['Potential/V'] >= oor_zeroc_potential)].copy().reset_index(drop=True)
    MOR__ = MOR[(MOR['Potential/V'] > mor_peak_potential) & (MOR['Current/A'] <= y_intercept) & (MOR['Potential/V'] <= mor_zeroc_potential)].copy().reset_index(drop=True)


    y_min = max(ORR__['Current/A'].min(), MOR__['Current/A'].min())

    # 在0~y_intercept之间取出1000个点,并求出其与ORR__和MOR__的交点,将Current/A, ORR_Potential/V, MOR_Potential/V存入dataframe
    cross_points = pd.DataFrame({'Current/A': np.linspace(y_min, y_intercept, 1000)})
    cross_points['ORR_Potential/V'] = cross_points['Current/A'].apply(lambda x: interpolate_potential(ORR__, x))
    cross_points['MOR_Potential/V'] = cross_points['Current/A'].apply(lambda x: interpolate_potential(MOR__, x))

    plt.fill_between(ORR__['Potential/V'], ORR__['Current/A']*1e6, y_intercept, color='grey', alpha=0.5)
    plt.fill_between(MOR__['Potential/V'], MOR__['Current/A']*1e6, y_intercept, color='grey', alpha=0.5)
    
    plt.axhline(y=0, color='r', linestyle='--', label=None)
    
    # 设置标签和图例
    plt.xlabel(r'Potential / V vs SCE')
    plt.ylabel(r'Current / $10^{-6}$A')
    plt.legend(frameon=False)

    # 设置网格线
    plt.gca().xaxis.set_major_formatter(FuncFormatter(format_x))
    plt.gca().yaxis.set_major_formatter(FuncFormatter(format_y))
    plt.gca().xaxis.set_minor_locator(ticker.AutoMinorLocator())
    plt.gca().yaxis.set_minor_locator(ticker.AutoMinorLocator())
    plt.grid(which='major', alpha=0.5)
    plt.grid(which='minor', alpha=0.2)

    # 显示图形
    plt.show()

    return cross_points

crosses = draw_ep(ep_ORR, ep_MOR)
\end{minted}

\begin{minted}[frame=lines, framesep=2mm, baselinestretch=1.2, fontsize=\footnotesize, linenos]{python}
def draw_UI(df):
    def format_y(value, tick_number):
        return float_to_sci(value)

    def format_x(value, tick_number):
        return f'{value:.3f}'

    colors = ['red', 'red', 'blue', 'blue', 'green', 'green', 'black', 'black']

    plt.rcParams['font.family'] = 'Times New Roman'
    plt.rcParams['mathtext.fontset'] = 'custom'
    plt.rcParams['mathtext.rm'] = 'Times New Roman'
    plt.rcParams['mathtext.it'] = 'Times New Roman:italic'
    plt.rcParams['mathtext.bf'] = 'Times New Roman:bold'
    plt.rcParams['font.size'] = 16
    plt.rcParams['axes.labelsize'] = 18
    plt.rcParams['xtick.labelsize'] = 16
    plt.rcParams['ytick.labelsize'] = 16

    plt.figure(figsize=(8, 6), dpi=1200)

    plt.plot(df['Potential/V'], df['Current/A']*1e6, 
                label=r'$U-I$', color=colors[1], linewidth=1.5)
    
    plt.xlabel(r'Potential / V vs SCE')
    plt.ylabel(r'Current / $10^{-6}$A')
    plt.legend(frameon=False)
    
    plt.gca().xaxis.set_major_formatter(FuncFormatter(format_x))
    plt.gca().yaxis.set_major_formatter(FuncFormatter(format_y))
    plt.gca().xaxis.set_minor_locator(ticker.AutoMinorLocator())
    plt.gca().yaxis.set_minor_locator(ticker.AutoMinorLocator())
    plt.grid(which='major', alpha=0.5)
    plt.grid(which='minor', alpha=0.2)

    plt.show()
    
crosses['Potential/V'] = crosses['MOR_Potential/V'] - crosses['ORR_Potential/V']
draw_UI(crosses)
\end{minted}

\begin{minted}[frame=lines, framesep=2mm, baselinestretch=1.2, fontsize=\footnotesize, linenos]{python}
crosses['Power/W'] = crosses['Potential/V'] * crosses['Current/A']

def draw_PU(df):
    def format_y(value, tick_number):
        return float_to_sci(value)

    def format_x(value, tick_number):
        return f'{value:.3f}'

    colors = ['red', 'red', 'blue', 'blue', 'green', 'green', 'black', 'black']

    plt.rcParams['font.family'] = 'Times New Roman'
    plt.rcParams['mathtext.fontset'] = 'custom'
    plt.rcParams['mathtext.rm'] = 'Times New Roman'
    plt.rcParams['mathtext.it'] = 'Times New Roman:italic'
    plt.rcParams['mathtext.bf'] = 'Times New Roman:bold'
    plt.rcParams['font.size'] = 16
    plt.rcParams['axes.labelsize'] = 18
    plt.rcParams['xtick.labelsize'] = 16
    plt.rcParams['ytick.labelsize'] = 16

    plt.figure(figsize=(8, 6), dpi=1200)

    plt.plot(df['Potential/V'], df['Power/W']*1e7,
                label=r'$P-U$', color=colors[1], linewidth=1.5)
    
    plt.xlabel(r'Potential / V vs SCE')
    plt.ylabel(r'Power / $10^{-7}$W')
    plt.legend(frameon=False)
    
    plt.gca().xaxis.set_major_formatter(FuncFormatter(format_x))
    plt.gca().yaxis.set_major_formatter(FuncFormatter(format_y))
    plt.gca().xaxis.set_minor_locator(ticker.AutoMinorLocator())
    plt.gca().yaxis.set_minor_locator(ticker.AutoMinorLocator())
    plt.grid(which='major', alpha=0.5)
    plt.grid(which='minor', alpha=0.2)

    plt.show()

draw_PU(crosses)
crosses.nlargest(1, 'Power/W')
\end{minted}
