\section{结果与讨论}

本实验采用三电极电解池对铂圆盘电极表面的电化学反应进行研究。首先,通过调整合适的电位窗口并在 $0.05\ \mathrm{M}$ 硫酸溶液中用氮气饱和进行电极活化,测定了不同扫描速度($0.5\ \mathrm{V/s}$、$0.2\ \mathrm{V/s}$ 和 $0.1\ \mathrm{V/s}$)下的循环伏安(CV)曲线。实验结果显示,随着扫描速度的增加,CV 曲线的电流峰值增大,而电压峰值变化不大。通过对 CV 曲线积分分析,计算得到扫速为分别为 $0.1,0.2,0.5\mathrm{~V\cdot s^{-1}}$ 时,铂电极表面单层氢原子吸附的电量分别为 $4.85,4.23,3.79 \times 10^{-6}\ \mathrm{C}$,从而推算出铂电极的电化学活性面积为 $2.31, 2.02, 1.81\ \mathrm{mm}^2$。

此外,本实验还测定了不同搅拌速度下铂电极表面氧还原反应(ORR)的 CV 曲线,并分析了搅拌条件对曲线形态的影响。在氧气饱和环境下,由于低电势区传质反应占主导,随着搅拌速率的增大,低电势区电流显著增大且曲线波动加剧,而高电势区几乎保持不变。同时,通过测定甲醇氧化反应(MOR)的 LSV 曲线,绘制了直接甲醇燃料电池(DMFC)的输出功率-输出电压曲线。结果表明,DMFC 电池的输出功率随电压的增大先增大后减小,输出功率在 $0.173\ \mathrm{V}$ 时达到最大值 $3.73 \times 10^{-7}\ \mathrm{W}$。

综上所述,本实验不仅成功测定了铂电极的电化学活性面积,而且深入探究了扫描速度和搅拌速率对循环伏安曲线的影响,为理解铂电极表面的电化学反应机理提供了重要的实验数据和理论支持。


