\section{结果与讨论}

\subsection{误差分析}

根据计算,各个参数的相对误差如下:

$$
\begin{aligned}
\frac{\sigma_\eta}{\eta} & =\frac{0.004 \times 10^{-3}}{1.083 \times 10^{-3}} \times 100 \% = 0.369 \% \\
\frac{\sigma_v}{v} & =\frac{0.00012}{0.00172} \times 100 \% = 6.977 \% \\
\frac{\sigma_l}{l} & =\frac{0.0022}{0.2248} \times 100 \% = 0.979 \% \\
\frac{\sigma_{\varphi}}{\varphi} & =\frac{1}{100} \times 100 \% = 1.000 \% \\
\frac{\sigma_{\varepsilon_r}}{\varepsilon_r} & =\frac{0.03}{80.10} \times 100 \% = 0.037 \%
\end{aligned}
$$

电动电势 \(\zeta\) 的相对偏差为:
$$
\frac{\sigma_\zeta}{\zeta}=\frac{0.0047}{0.05904} \times 100 \% = 7.125 \%
$$

由此可以看到,最主要的误差来源是界面迁移的平均速度 \(v\) 的测量,其次是两电极间距离 \(l\) 的测量与电泳仪电动电势 \(
\varphi\)。这表明在实验设计和测量中,应特别注意这几个参数尤其是界面迁移的平均速度 \(v\) 的精确度,以减少电动电势测量的总体误差。

电迁移速率的测量误差分为两部分:
\begin{enumerate}
    \item \textbf{线性拟合中斜率的不确定度:}
    在溶胶电泳法中,斜率的不确定度是电动电势 (\( \zeta \)) 测量的关键因素。主要的影响因素包括:

    \begin{enumerate}
        \item \textbf{界面位置的读数误差:} 在电泳实验中,观测并记录溶胶和辅助液之间的界面位置是关键步骤。由于溶胶溶液和辅助液的密度差异较小,界面往往不易明晰地观测到。特别是在正极区,由于溶胶会发生溶解,界面变得越来越模糊,从而增加了读数误差。尽管选择较为清晰的负极区域进行观测,但仍然存在显著的误差。
    
        \item \textbf{时间的读数误差:} 在测量电泳过程中,精确记录胶粒在电场中迁移的时间对于确定 \(\zeta\) 电势至关重要。时间的测量误差,无论是由于实验操作还是计时器的准确性问题,都会对最终的 \(\zeta\) 电势测量造成影响。
    \end{enumerate}
    
    因此,在进行溶胶电泳法测量时,需要尽可能减少界面位置和时间读数的误差,以提高 \(\zeta\) 电势测量的准确性和可靠性。
    
    \item \textbf{电泳法本身带来不确定度:}
    在溶胶电泳法测量电动电势 (\( \zeta \)) 时,实验条件的变化和溶胶的特性都可能导致结果的差异。主要的影响因素包括\cite{dong2013zeta}:

    \begin{enumerate}
        \item \textbf{实验条件的变异性:} 实验中的各种条件,如溶胶浓度、陈化时间、外加电压和电泳时间,都对 \( \zeta \) 电势的测量有显著影响。这些条件在教学实验中往往未能明确控制,导致实验结果的不一致性。由于这些因素很多是未知的,它们对误差的贡献虽然可能很大,但难以精确估计。
    
        \item \textbf{胶体制备过程的差异:} \( \zeta \) 电势与胶体的粒径、酸度、浓度等因素密切相关。胶体制备过程的微小差异可能对这些参数产生显著影响,从而导致不同样品间的 \( \zeta \) 电势值存在差异。每组实验样品的特性可能有所不同,进而影响实验结果的一致性。
    \end{enumerate}
    
    因此,在分析电泳法测量结果时,需要考虑这些因素对 \( \zeta \) 电势测量的潜在影响。为减少误差,建议在实验设计和执行过程中对这些变量进行严格控制和记录。
\end{enumerate}

\subsection{思考题}

\subsubsection{为什么加入\(\ce{FeCl3}\)的速度不宜太快}

在制备\(\ce{Fe(OH)3}\)胶体时,向热水中加入\(\ce{FeCl3}\)的速度不宜太快,主要是因为快速加入会导致\(\ce{FeCl3}\)在水中局部浓度过高,从而迅速形成大量的\(\ce{Fe(OH)3}\)沉淀,而不是形成稳定的胶体颗粒。这是因为\(\ce{FeCl3}\)水解生成\(\ce{Fe(OH)3}\)的反应如下:

\[
\ce{FeCl3 + 3H2O -> Fe(OH)3 + 3HCl}
\]

如果\(\ce{FeCl3}\)加得太快,会在局部区域迅速达到过饱和状态,导致\(\ce{Fe(OH)3}\)快速沉淀,无法形成较为均匀分散的胶体粒子。因此,为了获得较为均一和稳定的\(\ce{Fe(OH)3}\)胶体,应该缓慢并持续地向热水中加入\(\ce{FeCl3}\)溶液。

\subsubsection{电泳时正负极现象的详细解释}

在正极发生氧化反应,在负极发生还原反应,即对应着水的电解反应的半反应:

\[
\begin{gathered}
\ce{2H2O - 4e- -> O2 + 4H+}\\
\ce{2H2O + 2e- -> H2 + 2OH-} 
\end{gathered}
\]

正极反应产生的$ \ce{H+} $将 \(\ce{Fe(OH)3}\) 溶胶溶解,导致界面模糊;负极反应产生的$ \ce{OH-} $将 \(\ce{Fe(OH)3}\) 溶胶中带正电的溶胶离子沉淀,使得界面清晰的同时,会产生少量的絮状沉淀。

\subsection{实验的改进}

在本实验中的操作流程有着很大的改进空间。改进旨在简化操作流程,减少实验时间,提高粒径的重复性和实验的准确性。

\begin{enumerate}
    \item \textbf{溶胶的制备:} 传统的快速水解法简单但需要繁琐的渗析纯化,而胶溶法\cite{ruan2015hydroxide}降低了电导率,便于电泳实验和纯化处理,同时改善了粒径重复性和实验条件的控制。
    \item \textbf{溶胶的纯化:} 传统的热渗析法耗时长,需要频繁换水。为了优化纯化过程,可以使用酸/碱离子交换树脂法\cite{fan2009fe}。
    这种方法采用混合离子交换树脂(强酸、强碱型1:1)替代传统半透膜渗析,分为共同搅动法和树脂静止、凝胶流动两种方式。在共同搅动法中,将树脂倒入凝胶中,在磁力搅拌器下搅动,并实时监测电导率变化,直至满足电泳实验要求。这种方法操作简便、成本低廉,且不需要特定的仪器设备,非常适合教学实验。
    \item \textbf{获得清晰的界面:} 为了提高界面读数的精度,提出了改进方案,如使用更精细的注射器喷洒溶液。
\end{enumerate}

\subsection{结论}

本实验采用水解法和凝结法制备了 $\mathrm{Fe}(\mathrm{OH})_3$ 溶胶,并通过热渗析法进行纯化。实验中,对新制备的溶胶进行了 $191\ \mathrm{min}$ 的热渗析处理,使其电导率降至 $170.5\ \mu \mathrm{S/cm}$。同时,也使用了前期同学纯化好的溶胶(电导率为 $37.8\ \mu \mathrm{S/cm}$)进行电泳实验和聚沉实验。通过电泳法测定了经过纯化处理后的溶胶 $\zeta$ 电势为 $(59 \pm 5)\ \mathrm{mV}$。实验误差主要来源于电迁移速率的测量。本实验总结了各种误差源,并对实验方法提出了改进,如可以使用一种特殊的胶溶法\cite{ruan2015hydroxide}制备溶胶、使用酸/碱离子交换树脂法\cite{fan2009fe}纯化溶胶,使用注射器喷洒溶液的方法在电泳前获得清晰的辅助液-溶胶界面,为更准确地测定溶胶的电动电势提供了实验基础。
\\ 
\\

