\begin{titlepage}
% 页眉
\thispagestyle{plain}
% 校徽图片
\begin{figure}[h]
    \centering
    \includegraphics{pku.png}
\end{figure}
\vspace{24pt}
% 标题
\centerline{\zihao{-0} \textsf{物理化学实验报告}}
\vspace{40pt} % 空行
\begin{center}
    \begin{tabular}{cc}
        % 题目
        
        \addcell[2]{题目:\ } & \addcell[2]{氢氧化铁溶胶的制备及其性质研究} \\
        \cline{2-2}\\
        
    \end{tabular}
\end{center}
\vspace{20pt} % 空行
\begin{center}
    \doublespacing
    \begin{tabular}{cp{5cm}}
        % 姓名
        \addcell{姓\phantom{空格}名:\ } & \addcell{王子宸} \\
        \cline{2-2}
        % 学号
        \addcell{学\phantom{空格}号:\ } & \addcell{2100011873}\\
        \cline{2-2}
        % 组别
        \addcell{组\phantom{空格}别:\ } & \addcell{周四19组8号} \\
        \cline{2-2}
        % 实验日期
        \addcell{实验日期:\ } & \addcell{\zhdate{2023/11/30}}\\
        \cline{2-2}
        % 室温
        \addcell{温\phantom{空格}度:\ } & \addcell{\SI{17.0}{{}^\circ C}} \\
        \cline{2-2}
        % 大气压强
        \addcell{大气压强:\ } & \addcell{\SI{102.1}{kPa}}\\
        \cline{2-2}
    \end{tabular}
    \begin{tabular*}{\textwidth}{c}
    \\
    \\
        \hline % 分割线
    \end{tabular*}
\end{center}
% 摘要
\textsf{摘\ \ 要}\ \ 本实验采用水解法和凝结法制备了 $\mathrm{Fe}(\mathrm{OH})_3$ 溶胶,并通过热渗析法进行纯化。实验中,对新制备的溶胶进行了 $191\ \mathrm{min}$ 的热渗析处理,使其电导率降至 $170.5\ \mu \mathrm{S/cm}$。同时,也使用了前期同学纯化好的溶胶(电导率为 $37.8\ \mu \mathrm{S/cm}$)进行电泳实验和聚沉实验。通过电泳法测定了经过纯化处理后的溶胶 $\zeta$ 电势为 $(59 \pm 5)\ \mathrm{mV}$。实验误差主要来源于电迁移速率和电极间距离的测量。本实验总结了各种误差源,并提出了一些改进的实验方法,为更准确地测定溶胶的电动电势提供了实验基础。
\\ 
\\
% 关键字
\textsf{关键词}\ \ 物理化学实验;氢氧化铁溶胶;热渗析;电导;溶胶电泳;$\zeta$ 电势
\end{titlepage}

