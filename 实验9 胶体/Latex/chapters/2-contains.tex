\section{实验内容\cite{pcl2002}}

\subsection{仪器与药品}

$10\%$ $\ce{FeCl3}$ 溶液,$0.01\ \mathrm{mol/L}$ $\ce{AgNO3}$ 溶液,$0.1\ \mathrm{mol/L}$ $ \ce{KSCN} $溶液,$0.04\ \mathrm{mol/L}$ 和 $0.1\ \mathrm{mol/L}$ $\ce{KCl}$ 溶液,$0.002\ \mathrm{mol/L}$ $ \ce{K2SO4} $溶液,$0.001\ \mathrm{mol/L}$ $\ce{K3[Fe(CN)6]}$ 溶液。
 
滴管,单口烧瓶,烧杯,试管,$5\ \mathrm{mL}$ 量筒,透析袋及其夹子。

电加热套,电磁搅拌器,电导率仪,$\mathrm{U}$ 形电泳管,稳压电泳仪,铂电极,水浴锅,秒表。

\subsection{实验步骤与条件}

\subsubsection{\ce{Fe(OH)3} 溶胶的制备}
在 $250\ \mathrm{mL}$ 的洁净单口烧瓶中加入 $100\ \mathrm{mL}$ 去离子水和搅拌磁子。将烧瓶加热至水沸腾。在持续搅拌的同时,缓慢滴加 $5.0\ \mathrm{mL}$ 的 $10\%$ $\mathrm{FeCl}_3$ 溶液。滴加完成后,继续保持微沸状态并搅拌 $5\ \mathrm{min}$,得到红棕色的氢氧化铁溶胶。注意,此过程中不应补加水。

\subsubsection{\ce{Fe(OH)3} 溶胶的纯化}
先将透析袋剪裁成适当大小,并在去离子水中浸泡至变软。封闭透析袋一端,加入去离子水测试是否有漏液,并将其浸泡在去离子水中备用。

将制备好的氢氧化铁溶胶倒入透析袋中,并密封另一端,检查是否漏液。将封闭好的透析袋放入装有预热去离子水 $(60 \sim 80^{\circ}\mathrm{C})$ 的 $1000\ \mathrm{mL}$ 烧杯中,并在电磁搅拌下进行透析,尽量使透析袋转动以提高透析效率。

透析开始的一个小时内,每 $30\ \mathrm{min}$ 更换一次热水。使用 $\mathrm{KSCN}$ 溶液检测$ \ce{Fe^3+} $离子,$\ce{AgNO3}$ 溶液检测$ \ce{Cl-} $离子,直至$ \ce{Cl-} $离子浓度低于检测限。之后,用电导率仪监测透析速率,当电导率变化减缓时更换新的去离子水,直至透析液电导率低于 $30\ \mu\mathrm{S/cm}$。最后,用电导率仪测量溶胶的电导率,确保其低于 $200\ \mu\mathrm{S/cm}$。

将透析好的溶胶放入新换的去离子水中的烧杯内,并贴上标有电导值、姓名、学号的标签。用保鲜膜密封烧杯口,并将其放置于指定位置,供下周实验使用。

\subsection{溶胶电泳实验}

配制电导值与溶胶相同的 $\mathrm{KCl}$ 溶液作为辅助液。使用电导率仪测量得到溶胶与辅助液电导率均为 $37.8\ \mu\mathrm{S/cm}$。

电泳管应彻底清洗并确保其内壁无水珠附着。先用去离子水清洗 3 次,随后使用少量溶胶润洗 2 次。将电泳管垂直固定在支架上,并从中心部分倒入溶胶,直至液面接近 $4\ \mathrm{cm}$ 的刻度。

接着,使用滴管交替向电泳管的两端沿管壁缓慢滴加辅助液,直至液面高度达到 $8 \sim 9\ \mathrm{cm}$ 的刻度。初始阶段滴加速度应较慢,以防止液面振荡和界面模糊。若出现振荡,应暂停并等待液面稳定后再继续添加。液面较高时,可适度加快滴加速度。

将电极分别插入电泳仪的 “+” 和 “-” 插孔,并开启电泳仪预热。然后,将电极放入装有辅助液的烧杯中,调节电压至 $100 \pm 5\ \mathrm{V}$。之后,断开电路,将电极小心插入电泳管中,大约位于液面下 $1\ \mathrm{cm}$,确保正极在左侧管中,负极在右侧管中。记录下电极位置和初始界面位置。

实验开始后,每隔 $1\ \mathrm{min}$ 记录正极和负极的界面位置,至少测量 6 组数据。同时观察并记录界面状态以及电极表面的变化。实验结束后,关闭电源,并使用软尺测量电极间的实际距离,重复测量 3 次后取平均值,用以计算电动电势。




