%!TEX program = xelatex
\documentclass[cn,hazy,pku,12pt,normal,math=newtx,cite=super]{elegantnote}
\title{测量物理化学实验报告的写作焓}

\author{王子宸\quad210001873\\
19组\quad 8号}
\institute{化学与分子工程学院}

\expdate{\zhdate{2023/9/21}}
\temperature{298.2 \si{^{\circ}C}}
\pressure{100.61 \si{kPa}}

\usepackage{array}

\begin{document}

\maketitle

\keywords{国家精品课 \quad 物理化学实验 \quad 燃烧热的测定 \quad 雷诺图解法}

\abstracts{
(100字)
}

\newpage


\section{引言}

\subsection{实验目的}

\subsection{实验原理}

\subsection{实验方法}

% 简明扼要、逻辑严密、层次清楚

\subsection{实验}

\subsection{主要仪器与药品}

\begin{itemize}
    \item \textbf{仪器}:
    \item \textbf{药品}:
\end{itemize}
\subsection{实验步骤与条件}

% 可以直接贴预习报告照片,也可以重写

\section{数据处理与结果呈现}

% 根据实验原理的要求对原始实验数据进行适当的处理并作出符合规范的图或表;对所作图、表作出必要的描述和说明;依据图、表得出该实验待测量的最终运算结果(注意有效数字)

\section{结果与讨论}

% 对自己实验结果可靠性和可信度的论证和评价(如:误差的分析和估算、结果有效数字位数的确定等, 误差分析不要求面面俱到)
\subsection{误差分析}

% 对重要实验现象及所遇问题的分析和解释
\subsection{现象与解释}

% 结合物理化学原理、文献以及其它来源的证据和前人的结论,经严密推理得出符合逻辑的推论和结论
\subsection{推论与结论}

% 也可以简要总结一下该实验中的经验教训,或者提出对该实验进行改进的合理化建议。
\subsection{经验与建议}

\bibliography{reference}

\end{document}