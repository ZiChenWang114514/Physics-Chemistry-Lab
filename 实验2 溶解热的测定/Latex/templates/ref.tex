%!TEX program = xelatex
\documentclass[cn,hazy,pku,12pt,normal,math=newtx,cite=super]{elegantnote}
\title{测量物理化学实验报告的写作焓}

\author{王子宸\quad210001873\\
19组\quad 8号}
\institute{化学与分子工程学院}

\expdate{\zhdate{2023/9/21}}
\temperature{298.2 \si{^{\circ}C}}
\pressure{100.61 \si{kPa}}

\usepackage{array}

\begin{document}

\maketitle

\keywords{国家精品课 \quad 物理化学实验 \quad 燃烧热的测定 \quad 雷诺图解法}

\abstracts{

}

\newpage


\section{引言}

本模板在\href{https://github.com/Benzoin96485/PCLreport_elegant}{Github 仓库}上保持更新。

\section{实验部分}

本模板衍生自 ElegantNote 2.30 版本,它是基于标准的 \LaTeX{} 文类 article 重新设计的、格式更加简化的笔记模板。本中文模板的编译流程推荐使用 \hologo{XeLaTeX} - \hologo{biber} - \hologo{XeLaTeX} - \hologo{XeLaTeX}。

\subsection{纸张底色}

本模板内置护眼模式(\lstinline{mode=geye})和朦胧模式(\lstinline{mode=hazy})。其中护眼模式设置纸张底色为绿豆沙颜色,而朦胧模式为淡蓝色,开启的方法如下:
\begin{lstlisting}[frame=none]  
  \documentclass[geye]{elegantnote} % or
  \documentclass[mode=geye]{elegantnote}
  \documentclass[hazy]{elegantnote} % or
  \documentclass[mode=hazy]{elegantnote}
\end{lstlisting}

\begin{remark}
  如果你想为自己的文档添加底色,可以在导言区添加下面设置:
  \begin{lstlisting}
    \definecolor{geyecolor}{RGB}{199,237,204}
    \pagecolor{geyecolor}
  \end{lstlisting}
\end{remark}


\subsection{纸张大小}

原模板为了适配不同的设备,内置了 Pad(默认),Kindle,PC,A4 等纸张大小。这里为了符合实验报告要求,选择正常的 A4 大小的 PDF,即文档设置中的 \lstinline{device=normal}。页边距已经设置为 2.5 \si{cm}。


\subsection{数学字体选项}

本模板定义了一个数学字体选项(\lstinline{math}),可选项有三个:
\begin{enumerate}
  \item \lstinline{math=cm}(默认),使用 \LaTeX{} 默认数学字体(推荐,无需声明);
  \item \lstinline{math=newtx},使用 \lstinline{newtxmath} 设置数学字体(默认,潜在问题比较多)。
  \item \lstinline{math=mtpro2},使用 \lstinline{mtpro2} 宏包设置数学字体,要求用户已经成功安装此宏包。
\end{enumerate}


\subsection{中文字体选项}
模板提供中文字体选项 \lstinline{chinesefont},可选项有
\begin{enumerate}
\item \lstinline{ctexfont}:默认选项,使用 \lstinline{ctex} 宏包根据系统自行选择字体,可能存在字体缺失的问题,更多内容参考 \lstinline{ctex} 宏包\href{https://ctan.org/pkg/ctex}{官方文档}\footnote{可以使用命令提示符,输入 \lstinline{texdoc ctex} 调出本地 \lstinline{ctex} 宏包文档}。
\item \lstinline{founder}:方正字体选项,调用 \lstinline{ctex} 宏包并且使用 \lstinline{fontset=none} 选项,然后设置字体为方正四款免费字体,方正字体下载注意事项见后文。
\item \lstinline{nofont}:调用 \lstinline{ctex} 宏包并且使用 \lstinline{fontset=none} 选项,不设定中文字体,用户可以自行设置中文字体,具体见后文。
\end{enumerate}

\noindent \textbf{注意:} 使用 \lstinline{founder} 选项或者 \lstinline{nofont} 时,必须使用 \hologo{XeLaTeX} 进行编译。

\subsubsection{方正字体选项}
由于使用 \lstinline{ctex} 宏包默认调用系统已有的字体,部分系统字体缺失严重,因此,用户希望能够使用其它字体,我们推荐使用方正字体。方正的{\songti 方正书宋}、{\heiti 方正黑体}、{\kaishu 方正楷体}、{\fangsong 方正仿宋}四款字体均可免费试用,且可用于商业用途。用户可以自行从\href{http://www.foundertype.com/}{方正字体官网}下载此四款字体,在下载的时候请\textbf{务必}注意选择 GBK 字符集,也可以使用 \href{https://www.latexstudio.net/}{\LaTeX{} 工作室}提供的\href{https://pan.baidu.com/s/1BgbQM7LoinY7m8yeP25Y7Q}{方正字体,提取码为:njy9} 进行安装。安装时,{\kaishu Win 10 用户请右键选择为全部用户安装,否则会找不到字体。}

\begin{figure}[!htb]
\centering
\includegraphics[width=0.9\textwidth]{images/logo_pku.png}
\end{figure}

\subsubsection{其他中文字体}
如果你想完全自定义字体\footnote{这里仍然以方正字体为例。},你可以选择 \lstinline{chinesefont=nofont},然后在导言区设置
\begin{lstlisting}
\setCJKmainfont[BoldFont={FZHei-B01},ItalicFont={FZKai-Z03}]{FZShuSong-Z01}
\setCJKsansfont[BoldFont={FZHei-B01},ItalicFont={FZHei-B01}]{FZHei-B01}
\setCJKmonofont[BoldFont={FZHei-B01},ItalicFont={FZHei-B01}]{FZFangSong-Z02}
\setCJKfamilyfont{zhsong}{FZShuSong-Z01}
\setCJKfamilyfont{zhhei}{FZHei-B01}
\setCJKfamilyfont{zhkai}{FZKai-Z03}
\setCJKfamilyfont{zhfs}{FZFangSong-Z02}
\newcommand*{\songti}{\CJKfamily{zhsong}}
\newcommand*{\heiti}{\CJKfamily{zhhei}}
\newcommand*{\kaishu}{\CJKfamily{zhkai}}
\newcommand*{\fangsong}{\CJKfamily{zhfs}}
\end{lstlisting}


\subsection[颜色主题]{颜色主题\footnote{测试章节脚注。}}

本模板默认颜色主题为北大红,即 \textcolor{pku}{pku}。除此之外内置 5 套颜色主题,分别是 \textcolor{eblue}{blue}(默认),\textcolor{egreen}{green}, \textcolor{ecyan}{cyan}, \textcolor{sakura}{sakura} 和 \textcolor{black}{black}。如果不需要颜色,可以选择黑色(black)主题。颜色主题的设置方法:
\begin{lstlisting}[frame=none]  
  \documentclass[green]{elegantnote}
  \documentclass[color=green]{elegantnote}
  ...
  \documentclass[black]{elegantnote}
  \documentclass[color=black]{elegantnote}
\end{lstlisting}

\subsection{全局字号}
全局字体大小支持:8pt, 9pt, 10pt, 11pt, 12pt(我们所要求的小四), 14pt, 17pt 和 20pt;

\subsection{参考文献格式}
参考格式显示格式修改 \lstinline{cite} 可选为 \lstinline{authoryear}、\lstinline{numbers} (默认)和 \lstinline{super}。

\subsection{语言模式}

本模板在改为实验报告模板后,暂时只支持中文。


\subsection{定理类环境}

此模板采用了 \lstinline{amsthm} 中的定理样式,使用了 4 类定理样式,所包含的环境分别为
\begin{itemize}
  \item \textbf{定理类}:theorem,lemma,proposition,corollary;
  \item \textbf{定义类}:definition,conjecture,example;
  \item \textbf{备注类}:remark,note,case;
  \item \textbf{证明类}:proof。
\end{itemize}

\begin{remark}
在选用默认的 \lstinline{lang=cn} 时,定理类环境的引导词全部为中文,这些引导词也可以在 \lstinline{.cls} 文件中自行修改。
\end{remark}

\subsection{单位制}
此模板使用了 \lstinline{siunitx} 宏包,可以通过 \lstinline{\si} 等命令渲染规范的国际单位。


\section{数据与结果}

\begin{table}[htbp]
	\centering
	\small
	\caption{这是一个三线表的标题}
	\begin{tabular}{ccc}
		\toprule
		组别 &       物理量/单位       &        物理量/单位    \\
		\midrule
		$\cdots$ & $\cdots$ & $\cdots$ \\
		\bottomrule
	\end{tabular}%
	\label{tab:1}%
\end{table}%

根据表~\ref{tab:1},这个实验应该重做。


\begin{definition}[土豆]
	土豆就是跨院系选课开启时的北京大学选课网。
\end{definition}

\begin{example}
	2021年9月15日北京时间17:00的北京大学选课网是土豆。
\end{example}


\begin{figure}[htbp]
	\centering
	\includegraphics[width=0.7\textwidth]{images/logo_pku.png}
	
	\
	
	\caption{这是北京大学标识管理办公室提供的徽标}
	\label{fig:1}%
\end{figure}


\begin{theorem}[Hartree Fock 方程]\label{thm:HF}
对于无自旋、质量为 $m$ 的全同 $N$ 费米子体系,若 $V_1(\mathbf r)$ 是位矢为 $\mathbf r$ 的粒子所拥有的外场势能,$\varphi_n(\mathbf r)$ 为第 $n$ 个粒子的单粒子波函数,则 Hartree-Fock 方程为
\begin{equation}\label{eq:HF}
  \left[-\frac{\hbar^2}{2m}\Delta+V_1(\mathbf r)+V^{\text{dir}}(\mathbf r)\right]\varphi_n(\mathbf r)-\int\text d^3r'V_\text{ex}(\mathbf r,\mathbf r')\varphi_n(\mathbf r')=\tilde e_n\varphi_n(\mathbf r)
\end{equation}
\end{theorem}


根据定理~\ref{thm:HF},我们可以近似求解弱关联费米子体系在外场下的波函数。


\section{结果与讨论}

限于我们的水平,难免还存在错误和不当之处,恳请读者批评指正~\cite{pcl2002}。


\nocite{*}
\bibliography{reference}


\end{document}
