\section{结果与讨论}

\subsection{误差分析}

\subsubsection{实验操作误差}

本次实验中,实验操作误差主要来自于:
\begin{itemize}
    \item 比色皿没有严格垂直放置,光程长度无法保证一致性,造成误差;
    \item 对于透光率极低时(例如图 \ref{fig:5} 、图 \ref{fig:8}),会产生极大的信噪比,无法准确测定;
    \item 光源的强度难以保证不变,可能随时间、环境温度等因素变化,带来误差;
    \item 光路的位置有可能发生微小的变动,造成误差
\end{itemize}

\subsubsection{吸光度的选择}

根据 Lambert-Beer 定律, 有:
$$
A=-\lg T=\lg \frac{I_0}{I}=\epsilon b c
$$

考虑测量 $A$ 的相对误差 $E_r$ :
$$
E_r=\frac{\mathrm{d} A}{A}
$$

对 Lambert-Beer 定律两端微分, 得:
\begin{equation}\label{eq:5}
    \mathrm{d} A=-\lg e \frac{\mathrm{d} T}{T} \Longrightarrow E_r=\frac{\mathrm{d} A}{A}=0.434 \frac{\mathrm{d} T}{T \lg T}
\end{equation}

由公式 \eqref{eq:5} ,取$\mathrm{d}T=\pm 0.01$,可以求得在 $T=36.8 \%, A=0.434$ 时, 相对误差最小。

\subsubsection{Gaussian计算误差}

对于A3、A4与B1这样的柔性分子,其会有诸多构象, 不同构象对应的吸收光谱也不一样。计算吸收光谱时, 如果用的构象不是能量最低的构象, 这会造成和实验光谱产生很大的偏差。如果多个构象能量相差不太大, 而且彼此间在实验温度下足矣越过势垒能够相互转化, 那么必须计算各自的波尔兹曼分布对光谱做权重平均,绘制权重平均的光谱 \cite{sober14,sober15}。


\subsection{思考题}

\subsubsection{为何使用光栅的一级衍射}

在紫外可见吸收光谱的测定中,使用光栅的一级衍射是出于以下原因:
\begin{enumerate}
    \item \textbf{分辨率}:一级衍射光栅提供较好的分辨率。在衍射光栅中,分辨率随衍射级数的增加而提高,但是一级衍射已经能提供足够的分辨率,同时保持光强较高。
    \item \textbf{光强}:在一级衍射中,绝大部分的光强度都集中在这一级。更高级数的衍射虽然分辨率更高,但是它们会大幅度降低通过的光强度,这对于吸收光谱的灵敏度和准确度是不利的。
    \item \textbf{能量效率}:高阶衍射会减少通过特定波长的光量,这对检测系统的能量效率是不利的。一级衍射允许光谱设备使用较低的光源强度,这样就可以减少仪器的能耗,并且提高仪器的灵敏度。
    \item \textbf{色散特性}:光栅的色散特性依赖于衍射级数和入射光的波长。在一级衍射中,色散与波长的关系通常是线性的,这使得光谱的标定和解释更为直接和简单。
    \item \textbf{干扰降低}:使用一级衍射可以有效降低高阶衍射可能导致的干扰信号,这对于保证光谱纯度是重要的。
\end{enumerate}

\subsubsection{A组离子$\lambda_{\rm max}$与$\varepsilon$的变化}

A组离子的$\lambda_{\rm max}$与$\varepsilon$均会随着共轭体系的增长而显著增大。这是由于随着共轭体系的增长,能级间能量差减小,吸收光波长发生红移;同时,电子跃迁的概率(振子强度)也会随着共轭体系的增长而增加,较高的振子强度意味着体系对光的吸收更强,因此吸光系数会增加。

\subsubsection{如何取一维势阱长度}

\begin{itemize}
    \item 对于A组分子而言,一维势阱长度取$x+1$个C=C双键的共轭 + C-N单双键的共轭 + 2个苯环各1个双键共轭;
    \item 对于B1而言,没有探索到合适的一维势阱长度取值方法;
    \item 对于B2而言,取完整的共轭体系长度即可。
\end{itemize}

\subsubsection{二维势阱与三维势阱近似计算的体系,二维势阱的解}

\begin{itemize}
    \item \textbf{二维势阱}:量子点,f-色心等;
    \item \textbf{三维势阱}:石墨烯。
    \item \textbf{二维势阱的解}:
    \begin{equation*}
        E_{n, m}=\frac{\hbar^2 \pi^2}{2 m_e}\left(\frac{n^2}{a^2}+\frac{m^2}{b^2}\right)
    \end{equation*}
    其中,$E_{n, m}$为能级能量,$n, m$为$x, y$两个方向的量子数。
\end{itemize}

\subsubsection{有限深势阱能解释哪些物理现象与应用}

\begin{itemize}
    \item \textbf{量子点}:量子点是纳米级半导体粒子,其电子表现出有限深势阱模型的特性。量子点在特定条件下可以限制电子和空穴在三个空间维度中的运动,从而使其具有量子性质。这种现象被用于发光二极管(LEDs)和量子点太阳能电池中。
    \item \textbf{半导体器件}:在半导体物理学中,电子和空穴的运动可以通过有限深势阱来模拟。这对于设计和理解晶体管、激光器和其他电子元件的工作原理至关重要。
    \item \textbf{核物理}:在核物理中,有限深势阱模型可以用来描述原子核内的中子和质子,尤其是在考虑它们可能通过量子隧穿逃逸原子核时。
    \item \textbf{化学键}:分子中的化学键可以看作是电子在两个或更多原子核所形成的有限深势阱中的束缚状态。
    \item \textbf{超冷原子气体}:在超冷原子的物理学中,有限深势阱可用于对超低温下原子的束缚状态进行建模。
\end{itemize}