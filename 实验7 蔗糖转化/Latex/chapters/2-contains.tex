\section{实验内容\cite{pcl2002}}

\subsection{仪器与药品}

蔗糖 (AR), 盐酸 $(\mathrm{AR}, 2.96 \mathrm{~mol} / \mathrm{L}, 3.95 \mathrm{~mol} / \mathrm{L}, 5.85 \mathrm{~mol} / \mathrm{L})$ 。

旋光仪, 秒表, 恒温旋光管, 烧杯 $(500 \mathrm{~mL})$, 移液管 $(25 \mathrm{~mL}$ 若干), 磨口雉形瓶 $(100 \mathrm{~mL} \times 5)$, 量筒 $(100 \mathrm{~mL})$, 水浴装置, 电子台秤 $(0.01 \mathrm{~g})$ 。

\subsection{实验步骤与条件}

\subsubsection{旋光仪的使用和校零}

接通 WXG-4 目视旋光仪电源, 打开电源开关预热 $10 \mathrm{~min}$, 待完全发出钠黄光。洗净恒温旋光管, 连接好恒温水管路, 将旋光管加满去离子水, 并将管内气泡从加液口排净, 旋光管外壁残液用滤纸擦净、两端玻璃片用擦镜纸擦净, 放入旋光仪镜筒中。调节调焦螺旋,使视场中三分视场分界线最清晰。调节读盘转动手轮, 至三分视场消失, 视野暗度相同, 从读数放大盘中读出度盘的示数, 即为旋光仪零点的旋光度 $\alpha$ 。重复测量旋光仪零点 5 次, 取平均值作为旋光仪的零点。

\subsubsection{配制蔗糖溶液}

用粗天平称取 $30.03 \mathrm{~g}$ 蔗糖, 加入 $150.0 \mathrm{~mL}$ 蒸馏水, 在 $250 \mathrm{~mL}$ 烧杯中搅拌溶解。

\subsubsection{旋光度的测定}

用移液管移取 $25.00 \mathrm{~mL}$ 蔗糖溶液置于干燥的 $100 \mathrm{~mL}$ 雉形瓶中, 置于 $30^{\circ} \mathrm{C}$ 恒温水浴槽中预热。用另一支移液管移取 $25.00 \mathrm{~mL} 6.16 \mathrm{~M}$ 盐酸溶液($1\mathrm{~M}=1\mathrm{~mol\cdot L^{-1}}$), 移入装有蔗糖溶液的雉形瓶中。当酸流入一半时, 打开秒表开始计时。盐酸全部流入后迅速将混合液摇匀。取少量混合液润洗旋光管 $2 \sim 3$ 次, 用混合液装满旋光管。

用滤纸擦净管外壁的溶液, 尽快把旋光管放入旋光仪中, 测量不同时间 $t$ 时溶液的旋光角 $\alpha_t$ 。在反应开始 $15 \mathrm{~min}$ 内, 每半分钟到一分钟记录一次读数, 以后测量的时间间隔适

当加长, 测至旋光角 $\alpha_t$ 由右旋变为左旋, 至少获取 12 组有效数据。测量结束后, 将旋光管中的混合液倒回原雉形瓶中备用。

按照以上步骤, 依次使用 $3.14 \mathrm{~M}$、$4.19 \mathrm{~M}$、$6.16 \mathrm{~M}$ 的盐酸溶液, 进行混合液旋光度的测定。后续 4 组混合液不需倒回原雉形瓶中。()

\subsubsection{$\alpha_{\infty}$ 的测定}

将雉形瓶中保留备用的 $6.16 \mathrm{~M}$ 盐酸溶液与蔗糖溶液的混合液置于恒温水浴槽中预热,取少量混合液润洗旋光管 $2 \sim 3$ 次, 用混合液装满旋光管, 测定混合液的旋光度即为 $\alpha_{\infty}$ 。重复测量 5 次, 取平均值作为 $\alpha_{\infty}$ 的值。

