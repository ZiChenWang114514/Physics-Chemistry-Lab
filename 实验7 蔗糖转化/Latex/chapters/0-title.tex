\begin{titlepage}
% 页眉
\thispagestyle{plain}
% 校徽图片
\begin{figure}[h]
    \centering
    \includegraphics{pku.png}
\end{figure}
\vspace{40pt}
% 标题
\centerline{\zihao{-0} \textsf{物理化学实验报告}}
\vspace{20pt} % 空行
\begin{center}
    \begin{tabular}{cc}
        % 题目
        
        \addcell[2]{题目:\ } & \addcell[2]{测量蔗糖水解的化学反应动力学常数} \\
        \cline{2-2}\\
        
    \end{tabular}
\end{center}
\vspace{20pt} % 空行
\begin{center}
    \doublespacing
    \begin{tabular}{cp{5cm}}
        % 姓名
        \addcell{姓\phantom{空格}名:\ } & \addcell{王子宸} \\
        \cline{2-2}
        % 学号
        \addcell{学\phantom{空格}号:\ } & \addcell{2100011873}\\
        \cline{2-2}
        % 组别
        \addcell{组\phantom{空格}别:\ } & \addcell{周四19组8号} \\
        \cline{2-2}
        % 实验日期
        \addcell{实验日期:\ } & \addcell{\zhdate{2023/11/16}}\\
        \cline{2-2}
        % 室温
        \addcell{室\phantom{空格}温:\ } & \addcell{21.2\si{{}^\circ C}} \\
        \cline{2-2}
        % 大气压强
        \addcell{大气压强:\ } & \addcell{101.40 \si{kPa}}\\
        \cline{2-2}
    \end{tabular}
    \begin{tabular*}{\textwidth}{c}
    \\
        \hline % 分割线
    \end{tabular*}
\end{center}
% 摘要
\textsf{摘\ \ 要}\ \ 本实验通过测量蔗糖转化过程中体系旋光度的变化来研究蔗糖转化反应的化学反应动力学常数和氢离子的反应级数。实验结果表明,该反应为一级反应,通过绘制不同盐酸浓度下的 $\alpha_t-t$ 图和 $\ln\left(\alpha_t-\alpha_{\infty}\right)-t$ 图,计算得到反应速率常数 $k$ 和半衰期 $t_{1/2}$。加入的盐酸浓度分别为 $3.12 \mathrm{~mol\cdot L^{-1}}$、$4.19\mathrm{~mol\cdot L^{-1}}$、$6.16 \mathrm{~mol\cdot L^{-1}}$ 时,反应速率常数 $k$ 分别为$(7.96 \pm 0.14) \times 10^{-4} \mathrm{~L} \cdot \mathrm{mol}^{-1}$、$(1.31 \pm 0.03) \times 10^{-3} \mathrm{~L} \cdot \mathrm{mol}^{-1} $、$ (2.92 \pm 0.04) \times 10^{-3} \mathrm{~L} \cdot \mathrm{mol}^{-1}$,计算得到半衰期 $t_{1/2}$ 分别为 $(870 \pm 15) \mathrm{~s} $、$ (530 \pm 10) \mathrm{~s} $、$ (237 \pm 3) \mathrm{~s}$。此外,根据 $\ln\left(k/\mathrm{M}^{-1}\right)-\ln\left([\mathrm{H}^+]/\mathrm{M}\right)$ 图得出,蔗糖转化反应对氢离子的反应级数为二级。
\\
\\
% 关键字
\textsf{关键词}\ \ 物理化学实验;旋光仪;蔗糖;一级反应;速率常数;半衰期

\end{titlepage}



