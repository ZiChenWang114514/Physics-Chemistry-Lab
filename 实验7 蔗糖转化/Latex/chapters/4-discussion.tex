\section{结果与讨论}

\subsection{思考题}

\subsubsection{蔗糖的转化速率和哪些条件有关?}

蔗糖的转化速率主要与以下因素有关:

\begin{enumerate}
    \item \textbf{温度}:提高温度通常会加快转化速率。
    \item \textbf{浓度}:蔗糖和盐酸的浓度影响转化速率,浓度越高,速率通常越快。
    \item \textbf{压强}:在特定条件下,压强的改变可能影响转化速率。
    \item \textbf{催化剂}:盐酸在蔗糖转化反应中起催化作用,加速反应。
    \item \textbf{介质和物理形态}:反应介质和蔗糖的物理形态也会影响转化速率。
\end{enumerate}

\subsubsection{如何判断某一旋光物质是左旋还是右旋?}

判断某一旋光物质是左旋还是右旋的方法如下:

旋光物质是指能够旋转偏振平面的物质。在旋光仪中,光束通过旋光物质的样品时,其偏振平面会被旋转。旋光仪可以测量这种旋转,称为旋光角(\(\alpha\)),其表达式为:
\[ \alpha = [\alpha]_\lambda^T \cdot l \cdot c \]
其中,\([\alpha]_\lambda^T\) 是比旋光度,\(l\) 是样品的长度,\(c\) 是样品的浓度。

根据测量结果,我们可以判断物质的旋光性质:
\begin{itemize}
    \item 如果测得的旋光角 \(\alpha\) 为正值,则物质是右旋的。
    \item 如果测得的旋光角 \(\alpha\) 为负值,则物质是左旋的。
\end{itemize}

因此,通过测量旋光角的符号,我们可以判断物质是左旋还是右旋。这种方法适用于任何具有旋光性的物质,如某些有机化合物、糖类等。

\subsubsection{为什么配蔗糖溶液可用粗天平测量?蔗糖溶液如果不够,可另配制补充吗?}

在蔗糖转化反应中,该反应属于一级反应,即其速率仅与蔗糖的浓度成正比。由于实验的目的是测定反应速率常数 \(k\),而这个速率常数并不依赖于蔗糖的初始浓度。因此,在实验中,只需要大致控制蔗糖的初始浓度,而无需精确测量。

至于蔗糖溶液如果不够,不可以另行补充。原因在于,补充新的蔗糖溶液会改变测量时的蔗糖溶液浓度。虽然反应的速率常数 \(k\) 保持不变,但是这样的操作会改变最终旋光度的稳定值 \(\alpha_{\infty}\)。旋光度的变化是实验中用于计算速率常数的关键数据,任何改变初始条件都可能导致结果的不准确。

\subsubsection{一级反应的特点是什么?}

一级反应的特点总结如下:

\begin{enumerate}
    \item 反应速率与反应物的浓度成正比。在一级反应中,反应速率 \(v\) 与反应物浓度 \(c\) 的关系可以表示为 \(v = k \cdot c\),其中 \(k\) 为反应速率常数。
    \item 浓度的自然对数 \(\ln c\) 与时间 \(t\) 呈线性关系。这意味着随着时间的推移,反应物浓度的对数线性减少。
    \item 反应的半衰期与初始浓度无关,仅取决于反应速率常数 \(k\)。这表明无论反应物的起始浓度如何,达到其初始浓度一半所需的时间都是相同的。
\end{enumerate}

\subsubsection{已知蔗糖的 $[\alpha]_D^{20}=65.55^{\circ}$,设光源为钠光 $\mathrm{D}$ 线,旋光管长为 $20 \mathrm{~cm}$ 。试估算实验中所配的蔗糖和盐酸混合液的最初旋光角度是多少?}

$$
\alpha_0^{cal.}=[\alpha]_D^{20} \times \frac{l}{l^{\ominus}} \times \frac{m / V}{x_m^{\ominus}}=65.55^{\circ} \times \frac{20 \mathrm{~cm}}{10 \mathrm{~cm}} \times \frac{\frac{1}{2} \times 29.84 \mathrm{~g} / 150 \mathrm{~mL}}{1 \mathrm{~g} \cdot \mathrm{mL}}=13.04^{\circ}
$$

\begin{table}[H]
    \centering
    \bicaption{实验测定的初始旋光度}{The experimentally measured initial optical rotation}
    \begin{tabular}{ccc}
        \toprule
        $[\ce{H+}] / \si{M}$ & 拟合直线截距 $\lg \left(\alpha_0-\alpha_{\infty}\right)$ & $\alpha_0 /^{\circ}$ \\
        \midrule
        2.96 & $1.19 \pm 0.01$ & 11.5 \\
        3.95 & $1.21 \pm 0.01$ & 12.2 \\
        5.85 & $1.22 \pm 0.01$ & 12.6 \\
        \bottomrule
    \end{tabular}
    
    \label{tab:7}
\end{table}

$$
\alpha_0^{exp.}=\bar{\alpha}_0=12.1^{\circ}
$$

其与理论值较为接近,但仍有一定的差距。

\subsection{误差来源分析与实验改进}

本次实验主要存在两方面的误差来源:

\textbf{第一,人为误差}:由于人眼判断旋光仪读数位置时的不精确性,每次判断暗视场出现的位置略有差异,导致人为误差的产生。这主要体现在手动调节旋光仪度盘时,需要确保三分视场消失且视野暗度相同,但这一过程难以做到完全一致。

\textbf{第二,温度控制不足}:实验中的反应体系没有得到充分的恒温控制,特别是在加入盐酸溶液、摇匀、润洗旋光管等操作时。根据 Arrhenius 公式 $k = A e^{-\frac{E_a}{R T}}$ 可知,化学反应速率常数 $k$ 对温度极为敏感,即使是微小的温度变化也可能导致反应速率的显著改变。这些操作在室温下进行,可能会导致在该时间段内反应速率发生较大变化,从而对 $k$ 和 $t_{1 / 2}$ 的测定产生误差。

为了减少这些误差,可以采取以下措施:

\begin{enumerate}
    \item 使用自动化监视设备来判断旋光仪读数位置,从而减少人为误差的影响。
    \item 尽量缩短加入盐酸溶液、摇匀和润洗旋光管等操作所需的时间,并尽可能在恒温水浴中进行这些操作,以保证反应体系的温度稳定。
\end{enumerate}

\subsection{结论}

本实验通过测定蔗糖转化过程中的旋光度变化,深入探讨了蔗糖转化反应的化学动力学特性。实验结果验证了该反应符合一级反应的特征,并通过绘制不同盐酸浓度下的 $\alpha_t-t$ 曲线和 $\ln\left(\alpha_t-\alpha_{\infty}\right)-t$ 曲线,成功计算出反应速率常数 $k$ 和相应的半衰期 $t_{1/2}$。具体而言,当盐酸浓度为 $3.12 \mathrm{~mol\cdot L^{-1}}$、$4.19 \mathrm{~mol\cdot L^{-1}}$、$6.16 \mathrm{~mol\cdot L^{-1}}$ 时,相应的反应速率常数 $k$ 分别为 $(7.96 \pm 0.14) \times 10^{-4} \mathrm{~L} \cdot \mathrm{mol}^{-1}$、$(1.31 \pm 0.03) \times 10^{-3} \mathrm{~L} \cdot \mathrm{mol}^{-1} $、$(2.92 \pm 0.04) \times 10^{-3} \mathrm{~L} \cdot \mathrm{mol}^{-1}$,而计算所得的半衰期 $t_{1/2}$ 分别为 $(870 \pm 15) \mathrm{~s} $、$(530 \pm 10) \mathrm{~s} $、$(237 \pm 3) \mathrm{~s}$。

此外,通过分析 $\ln\left(k/\mathrm{M}^{-1}\right)-\ln\left([\mathrm{H}^+]/\mathrm{M}\right)$ 的关系图,本实验确认了蔗糖转化反应对氢离子的反应级数为二级。这一发现不仅丰富了对蔗糖转化反应动力学的理解,也为相关化学反应动力学实验提供了重要的实验数据和理论支持。
