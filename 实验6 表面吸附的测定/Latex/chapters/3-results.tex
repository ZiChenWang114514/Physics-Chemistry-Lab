\section{实验结果}

\subsection{溶液表面张力的测定}

\subsubsection{最大气泡压力法}

由于本人所使用的U型压力计,由于固定的原因,左右侧读数零点并无法保证严格的一致。因此,在每次测定前,均记录U型压力计左右侧的零点,记为$h_{0}$,记录一次;然后读取左右侧各自读取3组最大气泡压力时的示数,为$h_{1}$,取平均得到$\bar{h}_1$;分别计算两边的高度差,再相减得到总的高度差$\Delta h$,得到表 \ref{tab:1}:
\begin{equation}\label{eq:3}
    \Delta h = (h_{1,r} - h_{0,r}) - (h_{1,l} - h_{0, l})
\end{equation}


\begin{table}[htbp]
    \centering
    \bicaption{最大气泡压力法的高度差测量结果}{Measurement Results of Height Difference by Maximum Bubble Pressure Method}
    \begin{adjustwidth}{-2cm}{-2cm}
    \begin{center}
    \begin{tabular}{cccccccccccc}
    \toprule
    $c_{n\ce{BuOH}}$ & \multicolumn{5}{c}{左侧液面高度/\si{cm}} & \multicolumn{5}{c}{右侧液面高度/\si{cm}} & $\Delta h$ \\
     /\si{mol\cdot L^{-1}}& $h_{0,l}$ & $h_{1,l,1}$ & $h_{1,l,2}$ & $h_{1,l,3}$ & $\bar{h}_{1,l}$ & $h_{0,r}$ & $h_{1,r,1}$ & $h_{1,r,2}$ & $h_{1,r,3}$ & $\bar{h}_{1,r}$ & /\si{cm} \\
    \midrule
    0.0000 & 18.90 & 14.50 & 14.52 & 14.49 & 14.50 & 18.90 & 23.13 & 23.12 & 23.12 & 23.12 & 8.62 \\
    0.0218 & 18.90 & 14.85 & 14.86 & 14.87 & 14.86 & 18.90 & 22.83 & 22.84 & 22.85 & 22.84 & 7.98 \\
    0.0547 & 18.85 & 15.24 & 15.23 & 15.23 & 15.23 & 18.92 & 22.42 & 22.44 & 22.45 & 22.44 & 7.13 \\
    0.111 & 18.88 & 15.73 & 15.72 & 15.74 & 15.73 & 18.90 & 22.96 & 22.97 & 22.95 & 22.96 & 7.21 \\
    0.220 & 18.87 & 16.22 & 16.25 & 16.23 & 16.23 & 18.91 & 21.50 & 21.51 & 21.52 & 21.51 & 5.24 \\
    0.329 & 18.89 & 16.55 & 16.60 & 16.57 & 16.57 & 18.92 & 21.16 & 21.15 & 21.13 & 21.15 & 4.54 \\
    0.439 & 18.90 & 16.78 & 16.77 & 16.79 & 16.78 & 18.91 & 20.95 & 20.94 & 20.94 & 20.94 & 4.15 \\
    0.550 & 18.87 & 17.01 & 16.99 & 17.00 & 17.00 & 18.93 & 20.76 & 20.75 & 20.75 & 20.75 & 3.69 \\
    0.740 & 18.87 & 17.18 & 17.20 & 17.20 & 17.19 & 18.92 & 20.52 & 20.53 & 20.53 & 20.53 & 3.28 \\
    \bottomrule
    \end{tabular}
    \end{center}
    \end{adjustwidth}
    \label{tab:1}
\end{table}

以纯水作为标准参比,根据:
\begin{gather*}
    p = p^\ominus +\frac{2\gamma}{r} = p^\ominus +\rho g\Delta h \\
    \gamma \propto \Delta h
\end{gather*}

已知,在恒温\SI{30}{{}^\circ C}时,纯水的表面张力为\SI{71.18}{mN\cdot m^{-1}}\cite{haynes2016crc},因此,正丁醇水溶液的表面张力可以由公式 \eqref{eq:1} 计算。
\begin{equation}\label{eq:1}
    \gamma =\frac{\Delta h}{\Delta h_{\ce{H2O}}}\cdot \gamma_{\ce{H2O}}
\end{equation}

进一步,考虑高度测定中的误差,根据表 \ref{tab:1} 中,对于同一高度多次测量的结果,其最大偏差约为\SI{0.05}{cm},因此,不妨假设单次高度测量的读数误差为(在本实验报告中,为了方便起见,略去误差的单位,其单位与其所对应的物理量保持一致):
\begin{equation*}
    \sigma_h = \frac{0.05}{\sqrt{3}} = 0.029
\end{equation*}

根据 \eqref{eq:3},可以得到高度差的不确定度:
\begin{align*}
    \sigma_{\Delta h} &= \sqrt{3\times\left(\frac{\sigma_{h_{1,r}}}{3}\right)^2 + \left(-\sigma_{h_{0,r}}\right)^2 + \left(\sigma_{h_{0,l}}\right)^2  + 3\times\left(-\frac{\sigma_{h_{1,l}}}{3}\right)^2}\\
    &= \sqrt{\frac{8}{3}} \sigma_h = 0.047
\end{align*}
显然,高度差的不确定度与高度差本身的数值无关。

根据公式 \eqref{eq:1},可知:
\begin{align*}
\frac{\partial \gamma }{\partial \Delta h_{\ce{H2O}} }&=- \frac{\gamma_{\ce{H2O}} \Delta h}{\Delta h_{\ce{H2O}}^{2}} \\
\frac{\partial \gamma }{\partial \Delta h }&=\frac{\gamma_{\ce{H2O}}}{\Delta h_{\ce{H2O}}} \\
\sigma_\gamma &=\sqrt{\left(\frac{\partial \gamma }{\partial \Delta h_{\ce{H2O}} } \sigma_{\Delta h_{\ce{H2O}}}\right)^2+\left(\frac{\partial \gamma }{\partial \Delta h } \sigma_{\Delta h}\right)^2}\\
&= \sqrt{\left(- \frac{\gamma_{\ce{H2O}} \Delta h}{\Delta h_{\ce{H2O}}^{2}}\right)^2 + \left(\frac{\gamma_{\ce{H2O}}}{\Delta h_{\ce{H2O}}} \right)^2}\sigma_{\Delta h} \\
&= \frac{\sigma_{\Delta h}\gamma_{\ce{H2O}}}{\Delta h_{\ce{H2O}}}\sqrt{\frac{ \Delta h^2}{\Delta h_{\ce{H2O}}^{2}} + 1}
\end{align*}

最终,计算得到不同浓度正丁醇溶液的表面张力,如表 \ref{tab:2}。

\begin{table}[H]
    \centering
    \bicaption{最大气泡压力法测得的表面张力}{Surface Tension Measured by Maximum Bubble Pressure Method}
    \begin{tabular}{cccc}
    \toprule
    $c$/\si{mol\cdot L^{-1}} & $\ln \left(c_{n\ce{BuOH}}/\si{mol\cdot L^{-1}}\right)$ & $\Delta h$/\si{cm} & $\gamma$/\si{mN\cdot m^{-1}} \\
    \midrule
        0.0000 & $-$ & 8.62 & 71.18 \\
        0.0218 & -3.826 & 7.98 & 65.90 $\pm$ 0.53 \\
        0.0547 & -2.906 & 7.13 & 58.90 $\pm$ 0.50 \\
        0.111 & -2.198 & 6.21 & 51.28 $\pm$ 0.48 \\
        0.220 & -1.514 & 5.24 & 43.24 $\pm$ 0.45 \\
        0.329 & -1.112 & 4.54 & 37.52 $\pm$ 0.44 \\
        0.439 & -0.823 & 4.15 & 34.30 $\pm$ 0.43 \\
        0.550 & -0.598 & 3.69 & 30.50 $\pm$ 0.42 \\
        0.740 & -0.301 & 3.28 & 27.11 $\pm$ 0.42 \\
    \bottomrule
    \end{tabular}
    \label{tab:2}
\end{table}

\subsubsection{吊片法}

依次使用表面张力仪测量纯水与不同浓度的正丁醇水溶液的表面张力,记录得到表 \ref{tab:3}。由于仪器示数并不稳定,会存在较大范围的波动,因此,最终读数的确定有一定的随机性。

\begin{table}[H]
    \centering
    \bicaption{吊片法测得的表面张力}{Surface Tension Measured by the Wilhelmy Plate Method}
    \begin{tabular}{cccc}
    \toprule
    $c$/\si{mol\cdot L^{-1}} & $\ln \left(c_{n\ce{BuOH}}/\si{mol\cdot L^{-1}}\right)$ &  $\gamma$/\si{mN\cdot m^{-1}} & $T$/\si{{}^{\circ}C} \\
    \midrule
    0.0000 & $-$ & 71.72 & 22.9 \\
    0.0218 & -3.826 & 66.80 & 22.7 \\
    0.0547 & -2.906 & 61.20 & 22.8 \\
    0.111 & -2.198 & 54.56 & 23.1 \\
    0.220 & -1.514 & 48.40 & 22.9 \\
    0.329 & -1.112 & 42.20 & 23.1 \\
    0.439 & -0.823 & 37.90 & 23.1 \\
    0.550 & -0.598 & 34.45 & 23.0 \\
    0.740 & -0.301 & 30.55 & 22.9 \\
    \bottomrule
    \end{tabular}
    \label{tab:3}
\end{table}

\subsection{饱和吸附量与分子吸附面积的计算}

考虑真实界面 $\bar{G}^S=\bar{G}^S\left(T, P, A, n_i^S\right)$ 和理想的参考界面 $\bar{G}^R=\bar{G}^R\left(T, P, n_i^R\right)$,由于是理想界面,因此表面积对 $\bar{G}^R$ 影响,有:

$$
\begin{gathered}
\mathrm{d} \bar{G}^R=\left(\frac{\partial \bar{G}^R}{\partial T}\right) \mathrm{d} T+\left(\frac{\partial \bar{G}^R}{\partial P}\right) \mathrm{d} P+\sum_i\left(\frac{\partial \bar{G}^R}{\partial n_i^R}\right) \mathrm{d} n_i^R \\
\mathrm{~d} \bar{G}^S=\left(\frac{\partial \bar{G}^S}{\partial T}\right) \mathrm{d} T+\left(\frac{\partial \bar{G}^S}{\partial P}\right) \mathrm{d} P+\left(\frac{\partial \bar{G}^S}{\partial A}\right) \mathrm{d} A+\sum_i\left(\frac{\partial \bar{G}^S}{\partial n_i^S}\right) \mathrm{d} n_i^S
\end{gathered}
$$

令 $\dfrac{\partial \bar{G}^S}{\partial A}=\gamma$,$\dfrac{\partial \bar{G}^R}{\partial n_i^R}=\dfrac{\partial \bar{G}^S}{\partial n_i^S}=\bar{\mu}_i$,在恒温恒压条件下,有:
$$
\mathrm{d} \bar{G}^\sigma=\mathrm{d} \bar{G}^S-\mathrm{d} \bar{G}^R=\gamma \mathrm{d} A+\sum_i \bar{\mu}_i \mathrm{~d} n_i^\sigma
$$

由 Euler 定理可知:
$$
\bar{G}^\sigma=\gamma A+\sum_i \bar{\mu}_i n_i^\sigma
$$

对上式求全微分可得:
$$
\mathrm{d} \bar{G}^\sigma=\gamma \mathrm{d} A+A \mathrm{~d} \gamma+\sum_i \bar{\mu}_i \mathrm{~d} n_i^\sigma+\sum_i n_i^\sigma \mathrm{d} \bar{\mu}_i
$$

$\mathrm{d} \bar{G}^\sigma$ 的两式相减,则有:
$$
A \mathrm{~d} \gamma+\sum_i n_i^\sigma \mathrm{d} \bar{\mu}_i=0
$$

引人表面吸附量 $\Gamma_i=\dfrac{n_i^\sigma}{A}$,且理想溶液 $\mu=\mu^{\ominus}+R T \ln c$,则有:
\begin{equation}\label{eq:2}
    \Gamma=-\frac{1}{R T} \cdot \frac{\mathrm{d} \gamma}{\mathrm{d} \ln c}
\end{equation}

\subsubsection{最大气泡压力法}

使用表 \ref{tab:2} 的中,最大气泡压力法测定的表面张力$\gamma-\ln c_{n\ce{BuOH}}$作图,使用B-spline进行插值连线,并选取高浓度的$c_{n\ce{BuOH}}=0.15, 0.20, 0.30, 0.40, 0.45\:\si{mol\cdot L^{-1}}$,使用python scipy进行带$y$误差的线性回归,并使用python matplotlab绘图,选择用于线性拟合的点如表 \ref{tab:4},得到图 \ref{fig:1},回归直线的表达式为:
\begin{equation*}
    \frac{\gamma}{\si{mN\cdot m^{-1}}} = (-12.87 \pm 0.31)\ln\left(\frac{ c_{n\ce{BuOH}}}{\si{mol\cdot L^{-1}}}\right) + (23.57 \pm 0.40);\quad R^2 = 0.9984
\end{equation*}

\begin{figure}[htbp]
    \centering
    \includegraphics[width=.8\textwidth]{figures/1-1.png}
    \bicaption{最大气泡压力法:正丁醇水溶液 $\gamma-\ln c$ 等温曲线及拟合直线}{Maximum Bubble Pressure Method: $\gamma-\ln c$ isotherm and fitted curve}
    \label{fig:1}
\end{figure}

\begin{table}[htbp]
    \centering
    \bicaption{$\gamma-\ln c$ 曲线线性区取点相关数据}{Point data in the linear area of the $\gamma-\ln c$ curve}
    \begin{tabular}{cccc}
    \toprule
    $c / \mathrm{mol} \cdot \mathrm{L}^{-1}$ & $\ln \left(c / \mathrm{mol} \cdot \mathrm{L}^{-1}\right)$ & $\gamma / \mathrm{mN} \cdot \mathrm{m}^{-1}$ & $\sigma_\gamma$ \\
    \midrule
    0.150 & -1.897 & 47.96 & 0.48 \\
    0.200 & -1.609 & 44.51 & 0.48 \\
    0.300 & -1.204 & 38.65 & 0.45 \\
    0.400 & -0.916 & 35.45 & 0.44 \\
    0.450 & -0.799 & 33.94 & 0.43 \\
    \bottomrule
    \end{tabular}
    \label{tab:4}
\end{table}

根据公式 \eqref{eq:2},有
$$
\Gamma_{\infty}=-\frac{1}{R T} \cdot \frac{\mathrm{d} \gamma}{\mathrm{d} \ln c}=-\frac{-12.87 \times 10^{-3}}{8.314 \times 303.15} \mathrm{~mol} \cdot \mathrm{m}^{-2}=5.11 \times 10^{-6} \mathrm{~mol} \cdot \mathrm{m}^{-2}
$$

已知 $\sigma_{\frac{\mathrm{d} \gamma}{\mathrm{dln} c}}=0.31, \sigma_T=0.01$,则由误差传递公式,有:
$$
\begin{aligned}
\sigma_{\Gamma_{\infty}} & =\sqrt{\left(\frac{1}{R T} \cdot \frac{\sigma_T}{T}\right)^2+\left(\frac{1}{R T} \cdot \sigma_{\frac{\mathrm{d} \gamma}{\mathrm{dln} c}}\right)^2} \\
& =\frac{1}{8.314 \times 303.15} \sqrt{\left(\frac{0.01}{303.15}\right)^2+\left(0.31 \times 10^{-3}\right)^2} \\
& =0.12 \times 10^{-6} \mathrm{~mol} \cdot \mathrm{m}^{-2}
\end{aligned}
$$

最终可得,最大气泡法测得的饱和吸附量:
$$
\Gamma_{\infty}=(5.11 \pm 0.12) \times 10^{-6} \mathrm{~mol} \cdot \mathrm{m}^{-2}
$$

分子吸附面积:
$$
q = \frac{1}{N_{\mathrm{A}}\Gamma_{\infty}} = \frac{1}{5.11 \times 10^{-6} \times 6.022 \times 10^{23}} \times 10^{18} \mathrm{~nm}^2=0.325 \mathrm{~nm}^2
$$

其误差:
$$
\sigma_q=\sqrt{\left(\frac{\sigma_{\Gamma_{\infty}}}{\Gamma_{\infty}^2 N_A}\right)^2}=\sqrt{\left(\frac{0.12 \times 10^{-6}}{\left(5.11 \times 10^{-6}\right)^2 \times 6.022 \times 10^{23}} \times 10^{18}\right)^2}=0.0076 \mathrm{~nm}^2
$$

最终可得,最大气泡法测得的分子吸附面积:
$$
q = (0.325\pm 0.008) \mathrm{~nm}^2
$$

\subsubsection{吊片法}

由于吊片法测定表面张力时,读数波动过大,本人根据实际读数波动的范围估计其误差$\sigma_\gamma = \SI{1}{mN\cdot m^{-1}}$。与 3.2.1 类似,进行带$y$误差的线性回归,使用matplotlab作图,选择用于线性拟合的点如表 \ref{tab:5},得到图 \ref{fig:2},回归直线的表达式为:

\begin{equation*}
    \frac{\gamma}{\si{mN\cdot m^{-1}}} = (-13.70 \pm 0.62)\ln\left(\frac{ c_{n\ce{BuOH}}}{\si{mol\cdot L^{-1}}}\right) + (26.86 \pm 0.83);\quad R^2 = 0.9940
\end{equation*}

\begin{figure}[htbp]
    \centering
    \includegraphics[width=.8\textwidth]{figures/1-2.png}
    \bicaption{吊片法:正丁醇水溶液 $\gamma-\ln c$ 等温曲线及拟合直线}{Wilhelmy Plate Method: $\gamma-\ln c$ isotherm and fitted curve}
    \label{fig:2}
\end{figure}

\begin{table}[htbp]
    \centering
    \bicaption{$\gamma-\ln c$ 曲线线性区取点相关数据}{Point data in the linear area of the $\gamma-\ln c$ curve}
    \begin{tabular}{cccc}
    \toprule
    $c / \mathrm{mol} \cdot \mathrm{L}^{-1}$ & $\ln \left(c / \mathrm{mol} \cdot \mathrm{L}^{-1}\right)$ & $\gamma / \mathrm{mN} \cdot \mathrm{m}^{-1}$ & $\sigma_\gamma$ \\
    \midrule
    0.150 & -1.897 & 52.24 & 1 \\
    0.200 & -1.609 & 49.56 & 1 \\
    0.300 & -1.204 & 43.66 & 1 \\
    0.400 & -0.916 & 39.28 & 1 \\
    0.450 & -0.799 & 37.53 & 1 \\
    \bottomrule
    \end{tabular}
    \label{tab:5}
\end{table}

根据公式 \eqref{eq:2},有
$$
\Gamma_{\infty}=-\frac{1}{R T} \cdot \frac{\mathrm{d} \gamma}{\mathrm{d} \ln c}=-\frac{-13.70 \times 10^{-3}}{8.314 \times 303.15} \mathrm{~mol} \cdot \mathrm{m}^{-2}=5.44 \times 10^{-6} \mathrm{~mol} \cdot \mathrm{m}^{-2}
$$

已知 $\sigma_{\frac{\mathrm{d} \gamma}{\mathrm{dln} c}}=0.62, \sigma_T=0.01$,则由误差传递公式,有:
$$
\begin{aligned}
\sigma_{\Gamma_{\infty}} & =\sqrt{\left(\frac{1}{R T} \cdot \frac{\sigma_T}{T}\right)^2+\left(\frac{1}{R T} \cdot \sigma_{\frac{\mathrm{d} \gamma}{\mathrm{dln} c}}\right)^2} \\
& =\frac{1}{8.314 \times 303.15} \sqrt{\left(\frac{0.01}{303.15}\right)^2+\left(0.62 \times 10^{-3}\right)^2} \\
& =0.24 \times 10^{-6} \mathrm{~mol} \cdot \mathrm{m}^{-2}
\end{aligned}
$$

最终可得,吊片法测得的饱和吸附量:
$$
\Gamma_{\infty}=(5.44 \pm 0.24) \times 10^{-6} \mathrm{~mol} \cdot \mathrm{m}^{-2}
$$

分子吸附面积:
$$
q = \frac{1}{N_{\mathrm{A}}\Gamma_{\infty}} = \frac{1}{5.44 \times 10^{-6} \times 6.022 \times 10^{23}} \times 10^{18} \mathrm{~nm}^2=0.305 \mathrm{~nm}^2
$$

其误差:
$$
\sigma_q=\sqrt{\left(\frac{\sigma_{\Gamma_{\infty}}}{\Gamma_{\infty}^2 N_A}\right)^2}=\sqrt{\left(\frac{0.24 \times 10^{-6}}{\left(5.44 \times 10^{-6}\right)^2 \times 6.022 \times 10^{23}} \times 10^{18}\right)^2}=0.013 \mathrm{~nm}^2
$$

最终可得,吊片法测得的分子吸附面积:
$$
q = (0.305\pm 0.013) \mathrm{~nm}^2
$$

\subsection{饱和吸附时实际分子面积的计算}

因为 $\Gamma$ 实际上是一个过剩量,即使其等于 0(无吸附),表面上仍有溶质分子。故计算 $q$ 时忽略了表面上原有的溶质分子。对于较浓的溶液,在计算表面上溶质分子数时,除了吸附分子还应考虑原有分子。若 $\Gamma$ 以 $\mathrm{mol/m}^2$ 为单位,$c$ 以 $\mathrm{mol/dm}^3$ 为单位,$q_c$ 以 $\mathrm{nm}^2$ 为单位,则实际溶液浓度为 $c$ 时的吸附量为:
$$
q_c = \frac{10^{18}}{\Gamma N_A + 100 \times \left( c N_A \right)^{2/3}}
$$

考虑在高浓度时,饱和吸附时的实际分子面积:
\begin{equation}\label{eq:6}
    q_c = \frac{10^{18}}{\Gamma_\infty N_A + 100 \times \left( c N_A \right)^{2/3}}
\end{equation}


其误差可以表示为:
\begin{equation}\label{eq:7}
    \sigma_{q_c}=\sqrt{\left(\frac{N_A \cdot \sigma_{\Gamma_{\infty}}}{\left(\Gamma_{\infty} N_A+\left(c N_A\right)^{2 / 3}\right)^2}\right)^2+\left(\frac{2}{3} \cdot \frac{c^{-1 / 3} N_A^{2 / 3} \cdot \sigma_c}{\left(\Gamma_{\infty} N_A+\left(c N_A\right)^{2 / 3}\right)^2}\right)^2}
\end{equation}


根据公式 \eqref{eq:6}、\eqref{eq:7},可以计算在饱和吸附时,不同浓度的正丁醇溶液的实际分子面积及其不确定度,两种方法的计算结果分别如表 \ref{tab:6}。

\begin{table}[ht]
\centering
\bicaption{不同浓度下的$q_c$值}{Maximum Bubble Pressure Method: $q_c$ values at Different Concentrations}
\begin{tabular}{ccccc}
\toprule
\multirow{2}{*}{$c / \mathrm{mol \cdot L^{-1}}$} & \multicolumn{2}{c}{最大气泡压力法} & \multicolumn{2}{c}{吊片法} \\
& $\Gamma_\infty/\si{mol\cdot m^{-2}}$ & $q_c / \mathrm{nm^2}$ & $\Gamma_\infty/\si{mol\cdot m^{-2}}$ & $q_c / \mathrm{nm^2}$ \\
\midrule
0.220 & \multirow{5}{*}{0.511} & 0.300 $\pm$ 0.008 & \multirow{5}{*}{0.544} & 0.283 $\pm$ 0.013\\
0.329 & & 0.293 $\pm$ 0.008 & & 0.277 $\pm$ 0.013\\
0.439 & & 0.287 $\pm$ 0.008 & & 0.271 $\pm$ 0.013\\
0.550 & & 0.281 $\pm$ 0.008 & & 0.266 $\pm$ 0.013\\
0.740 & & 0.273 $\pm$ 0.008 & & 0.259 $\pm$ 0.013\\
\bottomrule
\end{tabular}
\label{tab:6}
\end{table}










