\section{结果与讨论}

\subsection{与文献值的比较}

查阅 CRC 手册 \cite{haynes2016crc} 可知,在冰浴下,正丁醇浓度 $c = 0.854 \, \mathrm{mol/L}$ 时,$q_c = 0.237 \, \mathrm{nm}^2$。

本实验中,最大气泡压力法测得 $\Gamma_{\infty}^{30^\circ \mathrm{C}} = (5.11 \pm 0.12) \times 10^{-6} \, \mathrm{mol \cdot m^{-2}}$,吊片法测得 $\Gamma_{\infty}^{23^\circ \mathrm{C}} = (5.44 \pm 0.24) \times 10^{-6} \, \mathrm{mol \cdot m^{-2}}$。考虑到温度的影响,近似认为冰浴时为 $T = 273.15 \, \mathrm{K}$,忽略温度对表面张力的影响,近似有公式:
$$
\Gamma_{\infty}^{T_1} = \frac{T_2}{T_1} \times \Gamma_{\infty}^{T_2}
$$
则有最大气泡压力法测得 $\Gamma_{\infty}^{0^\circ \mathrm{C}} = (5.67 \pm 0.13) \times 10^{-6} \, \mathrm{mol \cdot m^{-2}}$,吊片法测得 $\Gamma_{\infty}^{0^\circ \mathrm{C}} = (5.90 \pm 0.26) \times 10^{-6} \, \mathrm{mol \cdot m^{-2}}$。

分别将这两个值和 $c = 0.854 \, \mathrm{mol/L}$ 代入到公式 \eqref{eq:6} 与 \eqref{eq:7} 中,可得:最大气泡压力法 $q_c = (0.247 \pm 0.007) \, \mathrm{nm}^2$;吊片法 $q_c = (0.238 \pm 0.012) \, \mathrm{nm}^2$。这两个值在误差范围内都与文献值非常接近,考虑到实验过程中进行了较多的近似,这样的精度是完全可以接受的。

\subsection{结论}

本实验测量了正丁醇水溶液的表面吸附量和表面吸附分子的横截面积。利用最大气泡压力法和吊片法, 来测量不同浓度待测样品的表面张力, 利用 Gibbs 吸附等温式通过拟合高浓度线性部分的 $\gamma-\ln c$ 图, 来计算正丁醇水溶液的饱和吸附量为: 最大气泡压力法 $\Gamma_{\infty}^{30^{\circ} \mathrm{C}}=(5.11 \pm 0.12) \times 10^{-6} \mathrm{~mol} \cdot \mathrm{m}^{-2}$; 吊片法 $\Gamma_{\infty}^{23^{\circ} \mathrm{C}}=(5.44 \pm 0.24) \times 10^{-6} \mathrm{~mol} \cdot \mathrm{m}^{-2}$ 。进一步地, 计算表面吸附分子的横截面积为: 最大气泡压力法 $q=(0.325 \pm 0.008) \mathrm{~nm}^2$;吊片法 $q=(0.305 \pm 0.013) \mathrm{~nm}^2$ 。最后, 讨论了考虑溶剂分子的实际横截面积 $q_c$ 在不同正丁醇浓度时的取值,得出了与文献值非常接近的结果。

