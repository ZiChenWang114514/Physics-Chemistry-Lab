\begin{titlepage}
% 页眉
\thispagestyle{plain}
% 校徽图片
\begin{figure}[h]
    \centering
    \includegraphics{pku.png}
\end{figure}
\vspace{24pt}
% 标题
\centerline{\zihao{-0} \textsf{物理化学实验报告}}
\vspace{40pt} % 空行
\begin{center}
    \begin{tabular}{cc}
        % 题目
        
        \addcell[2]{题目:\ } & \addcell[2]{测量正丁醇水溶液的表面张力与表面吸附量} \\
        \cline{2-2}\\
        
    \end{tabular}
\end{center}
\vspace{20pt} % 空行
\begin{center}
    \doublespacing
    \begin{tabular}{cp{5cm}}
        % 姓名
        \addcell{姓\phantom{空格}名:\ } & \addcell{王子宸} \\
        \cline{2-2}
        % 学号
        \addcell{学\phantom{空格}号:\ } & \addcell{2100011873}\\
        \cline{2-2}
        % 组别
        \addcell{组\phantom{空格}别:\ } & \addcell{周四19组8号} \\
        \cline{2-2}
        % 实验日期
        \addcell{实验日期:\ } & \addcell{\zhdate{2023/11/9}}\\
        \cline{2-2}
        % 室温
        \addcell{室\phantom{空格}温:\ } & \addcell{19.8\si{{}^\circ C}} \\
        \cline{2-2}
        % 大气压强
        \addcell{大气压强:\ } & \addcell{100.95 \si{kPa}}\\
        \cline{2-2}
    \end{tabular}
    \begin{tabular*}{\textwidth}{c}
    \\
    \\
        \hline % 分割线
    \end{tabular*}
\end{center}
% 摘要
\textsf{摘\ \ 要}\ \ 本实验测量了正丁醇水溶液的表面吸附量和表面吸附分子的横截面积。利用最大气泡压力法和吊片法, 来测量不同浓度待测样品的表面张力, 利用 Gibbs 吸附等温式通过拟合高浓度线性部分的 $\gamma-\ln c$ 图, 来计算正丁醇水溶液的饱和吸附量为: 最大气泡压力法 $\Gamma_{\infty}^{30^{\circ} \mathrm{C}}=(5.11 \pm 0.12) \times 10^{-6} \mathrm{~mol} \cdot \mathrm{m}^{-2}$; 吊片法 $\Gamma_{\infty}^{23^{\circ} \mathrm{C}}=(5.44 \pm 0.24) \times 10^{-6} \mathrm{~mol} \cdot \mathrm{m}^{-2}$ 。进一步地, 计算表面吸附分子的横截面积为: 最大气泡压力法 $q=(0.325 \pm 0.008) \mathrm{~nm}^2$;吊片法 $q=(0.305 \pm 0.013) \mathrm{~nm}^2$ 。最后, 讨论了考虑溶剂分子的横截面积 $q_c$ 。
\\
\\
% 关键字
\textsf{关键词}\ \ 物理化学实验;表面吸附;表面张力;正丁醇水溶液;最大气泡压力法;吊片法
\end{titlepage}