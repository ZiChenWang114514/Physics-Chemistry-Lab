\section{结果与讨论}

\subsection{误差分析}

\subsubsection{实验结果与文献值的偏差}

根据文献\cite{haynes2016crc},在压强\SI{102.26}{kPa}下,乙醇-环己烷体系的最低恒沸点为\SI{64.80}{\celsius},此
时乙醇的摩尔分数为0.4540,则有:

$$
\begin{aligned}
\chi_{\ce{EtOH}} & =\frac{m_{\ce{EtOH}}}{m_{\text {tot. }}} \\
& =\frac{M_{\ce{EtOH}} x_{\ce{EtOH}}}{M_{\ce{EtOH}} x_{\ce{EtOH}}+M_{\ce{CyH}}\left(1-x_{\ce{EtOH}}\right)} \\
& =\frac{46.07 \times 0.4540}{46.07 \times 0.4540+84.16 \times(1-0.4540)} \\
& =0.3128
\end{aligned}
$$

本实验实际测定时的气压为\SI{102.21}{kPa},与文献值的偏差小于千分之一,可以认为实验气压近似等于\SI{102.26}{kPa}。

因此,可以得到乙醇-环己烷体系最低恒沸点的相对误差:
\[
\sigma_T = \frac{65.49-64.80}{64.80} = 1.06\%
\]
最低恒沸点对应乙醇质量分数的相对误差:
\[
\sigma_{\chi_{\rm{EtOH}}} = \frac{0.3171-0.3128}{0.3128}=1.37\%
\]

由此可见,本实验测定得到的乙醇-环己烷体系的最低恒沸点与文献值很接近,证明实验相对理想。

\subsubsection{实验的误差来源}

由于最低恒沸点的选取涉及到B-spline插值的问题、求导、取交点的问题,难以显式地推导误差的传递。因此,不妨讨论在最低恒沸点时,由标准工作曲线计算乙醇质量分数所带来的误差。

考虑使用二次拟合的曲线公式 \eqref{eq:1},为了更方便的估计在最低恒沸点附近,使用工作曲线反向求解得到的误差,我们选取气相线的折射率最低点$n=1.3963$,已知折射率 $n$ 的情况下,得到乙醇-环己烷体系的乙醇质量分数 $\chi_{\rm{EtOH}}$:(以下内容全部由Python SymPy库的自动求导函数求出,并自动转化为Latex表达式,因此在过程中会有一些繁琐的步骤)
\begin{equation*}
\begin{aligned}
    \chi_{\rm{EtOH}}&=\frac{- b - \left(- 4 a \left(c - n\right) + b^{2}\right)^{0.5}}{2 a}\\
    &=\frac{- \left(-0.07888\right) - \left(- 4 \times \left(0.01371\right) \times \left(\left(1.4208\right) - \left(1.3963\right)\right) + \left(-0.07888\right)^{2}\right)^{0.5}}{2 \times \left(0.01371\right)}\\
    &=0.3295\ 
\end{aligned}
\end{equation*}

求偏导数:
\begin{equation*}
\begin{aligned}
\frac{\partial \chi_{\rm{EtOH}} }{\partial a }&=\frac{a \left(c - n\right) + \dfrac{\left(b + \left(- 4 a \left(c - n\right) + b^{2}\right)^{0.5}\right) \left(- 4 a \left(c - n\right) + b^{2}\right)^{0.5}}{2}}{a^{2} \left(- 4 a \left(c - n\right) + b^{2}\right)^{0.5}}\\
&=\left[\left(0.01371\right) \times \left(\left(1.4208\right) - \left(1.3963\right)\right) + \right.\\
&\quad\frac{1}{2}\times\left(\left(-0.07888\right) + \left(- 4 \times \left(0.01371\right) \times \left(\left(1.4208\right) - \left(1.3963\right)\right) + \left(-0.07888\right)^{2}\right)^{0.5}\right) \\
&\quad\times\left. \left(- 4 \times \left(0.01371\right) \times \left(\left(1.4208\right) - \left(1.3963\right)\right) + \left(-0.07888\right)^{2}\right)^{0.5}\right]\\
&\quad\left[{\left(0.01371\right)^{2} \times \left(- 4 \times \left(0.01371\right) \times \left(\left(1.4208\right) - \left(1.3963\right)\right) + \left(-0.07888\right)^{2}\right)^{0.5}}\right]^{-1}\\
&=1.6\\ 
\frac{\partial \chi_{\rm{EtOH}} }{\partial b }&=- \frac{0.5 b}{a \left(- 4 a c + 4 a n + b^{2}\right)^{0.5}} - \frac{1}{2 a}\\
&=- \frac{0.5 \times \left(-0.07888\right)}{\left(0.01371\right) \times \left(- 4 \times \left(0.01371\right) \times \left(1.4208\right) + 4 \times \left(0.01371\right) \times \left(1.3963\right) + \left(-0.07888\right)^{2}\right)^{0.5}} -\\
&\quad\frac{1}{2 \times \left(0.01371\right)}=4.7\\
\frac{\partial \chi_{\rm{EtOH}} }{\partial c }
&=\frac{1.0}{\left(- 4 a \left(c - n\right) + b^{2}\right)^{0.5}}\\
&=\frac{1.0}{\left(- 4 \times \left(0.01371\right) \times \left(\left(1.4208\right) - \left(1.3963\right)\right) + \left(-0.07888\right)^{2}\right)^{0.5}}=14.0\\
\frac{\partial \chi_{\rm{EtOH}} }{\partial n }&=- \frac{1.0}{\left(- 4 a \left(c - n\right) + b^{2}\right)^{0.5}}\\
&=- \frac{1.0}{\left(- 4 \times \left(0.01371\right) \times \left(\left(1.4208\right) - \left(1.3963\right)\right) + \left(-0.07888\right)^{2}\right)^{0.5}}=-14.0\\
\end{aligned}
\end{equation*}

计算误差:
\begin{equation*}
\begin{aligned}
\sigma_{\chi_{\rm{EtOH}}}&=\sqrt{\left(\frac{\partial \chi_{\rm{EtOH}} }{\partial a } \sigma_{a}\right)^2+\left(\frac{\partial \chi_{\rm{EtOH}} }{\partial b } \sigma_{b}\right)^2+\left(\frac{\partial \chi_{\rm{EtOH}} }{\partial c } \sigma_{c}\right)^2+\left(\frac{\partial \chi_{\rm{EtOH}} }{\partial n } \sigma_{n}\right)^2}\\
&=\sqrt{\left(1.6 \times 0.0026\right)^2+\left(4.7 \times 0.0027\right)^2+\left(14.0 \times 0.0006\right)^2+\left(-14.0 \times 0.0003\right)^2}\\
&=\sqrt{\left(0.004\right)^2+\left(0.013\right)^2+\left(0.0086\right)^2+\left(-0.0043\right)^2}\\
&=0.016\
\end{aligned}
\end{equation*}

最终,得到乙醇-环己烷体系的乙醇质量分数 $\chi_{\rm{EtOH}}$:
\begin{equation*}
\chi_{\rm{EtOH}}=\left (0.330 \pm 0.016 \right )\
\end{equation*}

由此可见,在最低恒沸点附近,使用工作曲线反向解出的乙醇质量分数 $\chi_{\rm{EtOH}}$ 的误差约为 $0.0016$,不妨假设最低恒沸点处的误差与之相同。

由此可见,误差的主要来源是线性拟合中的一次项,也即工作曲线本身的拟合误差。

\subsection{思考题}

\subsubsection{本实验中,气液两相是如何达到平衡的?}

在使用恒沸点仪测定乙醇-环己烷二元气液相图时,气液两相达到平衡的过程通常遵循以下步骤:

\begin{enumerate}
    \item 加热和蒸发:混合液体被加热,其中一部分蒸发成气体。这个过程开始时,气体和液体的组成可能不同。
    \item 再凝结和回流:蒸发成气体的混合物被冷凝成液体,然后返回到混合物中。这个过程称为回流。
    \item 达到平衡:通过不断的蒸发和回流,气体和液体的组成逐渐接近一致,最终达到相平衡。在相平衡状态下,气体和液体的组成在给定的温度和压力下保持恒定。
\end{enumerate}

\subsubsection{冷凝管$D$处的体积太大,会有怎样的后果?}

如果在支管与冷凝管处的体积太大,可能会有以下后果:

\begin{enumerate}
    \item 延迟达到平衡:较大的体积意味着更多的液体可以蒸发并进入冷凝管。这可能导致达到真正的气液平衡状态需要更长的时间,因为需要更多的物质在气液之间转移。
    \item 组成分布不均:在较大的体积中,可能会出现组成分布不均匀的情况。例如,在某些区域,蒸汽可能会富集一种组分,而在其他区域则可能富集另一种组分。这可以影响最终测量的准确性。
    \item 热量损失:较大的体积也可能导致更多的热量损失,影响系统的热平衡,进而影响平衡温度的测定。
\end{enumerate}

\subsubsection{平衡时气-液两相温度应不应该一样?实际是否一样?如何防止温度的差异?}

在理想情况下,气-液两相平衡时的温度应该是相同的。这是因为相平衡是指在一定温度和压力下,气体和液体的化学势达到相等,从而没有物质的净迁移。在这种状态下,气体和液体应处于同一温度。

然而,在实际操作中,气-液两相的温度可能会有所不同,原因可能包括:

\begin{enumerate}
    \item 系统热不均匀:在实验过程中,由于加热方式、热传导和热对流的影响,系统中可能存在温度梯度。
    \item 冷凝和蒸发过程:蒸发是吸热过程,而冷凝是放热过程。如果冷凝和蒸发的热量没有被完全交换,这可能导致气相和液相的温度不一致。
\end{enumerate}

为了尽量减少气-液两相温度的差异,可以采取以下措施:

\begin{enumerate}
    \item 优化加热系统,减少热损失:使用均匀加热的设备,确保整个系统的温度分布尽可能均匀,通过隔热材料和设计来减少热量的外泄,保持系统的热稳定性。在本实验中,即使用了棉线作为隔热材料,包裹玻璃仪器,以减少热交换及其带来的热损失。
    \item 提高热交换效率:使用有效的热交换装置,如高效冷凝器,确保蒸发和冷凝过程中的热量能够有效交换。本实验中,使用电阻丝加热,电阻丝的热交换效率较高。
    \item 实时监控温度:在气相和液相中均设置温度传感器,实时监测温度差异,并根据需要调整实验条件。在本实验中,有且仅有一个温度传感器,这显然是不够的,应该增加温度传感器的个数,以实现更好的温度监控。
\end{enumerate}

\subsection{实验的问题与改进}

实际实验中,由于加热用的电阻丝位于蒸馏烧瓶底部,且整个系统保温较差,蒸馏烧瓶内存在自下而上的温度梯度,靠近电阻丝的部分温度较高,而远离电阻丝的冷凝管处温度显著较低,且冷凝管内也存在自下而上的温度梯度,产生了显著的分馏现象,从而使得所测得的液相组分并非测量温度下的实际组分,造成了一定的实验误差,可能因此导致了恒沸混合物 $\chi_{\rm{EtOH}}$ 与文献参考值的偏离。

解决方法:使用恒温水浴槽将体系浸没,或在体系外包裹更加有效的保温隔热材料,以消除体系内部的温度梯度\cite{song2020temperature}。

\subsection{实验总结}

本实验测量了阿贝折射仪的工作曲线,并通过测量不同质量分数的乙醇-环已烷恒沸体系气相和液相的折射率从而确定其质量分数,绘制了乙醇-环已烷双液体系的相图,求得这个体系的最低恒沸点为 $65.49^{\circ} \mathrm{C}$,对应的乙醇的质量分数为 $0.3171$,与文献值基本接近,主要的误差来源为工作曲线的测定与拟合。

