\begin{titlepage}
% 页眉
\thispagestyle{plain}
% 校徽图片
\begin{figure}[h]
    \centering
    \includegraphics{pku.png}
\end{figure}
\vspace{24pt}
% 标题
\centerline{\zihao{-0} \textsf{物理化学实验报告}}
\vspace{40pt} % 空行
\begin{center}
    \begin{tabular}{cc}
        % 题目
        
        \addcell[2]{题目:\ } & \addcell[2]{液体饱和蒸气压的测定} \\
        \cline{2-2}\\
        
    \end{tabular}
\end{center}
\vspace{12pt} % 空行
\begin{center}
    \doublespacing
    \begin{tabular}{cp{5cm}}
        % 姓名
        \addcell{姓\phantom{空格}名:\ } & \addcell{王子宸} \\
        \cline{2-2}
        % 学号
        \addcell{学\phantom{空格}号:\ } & \addcell{2100011873}\\
        \cline{2-2}
        % 组别
        \addcell{组\phantom{空格}别:\ } & \addcell{周四19组8号} \\
        \cline{2-2}
        % 实验日期
        \addcell{实验日期:\ } & \addcell{\zhdate{2023/12/7}}\\
        \cline{2-2}
        % 室温
        \addcell{温\phantom{空格}度:\ } & \addcell{\SI{19.30}{{}^\circ C}} \\
        \cline{2-2}
        % 大气压强
        \addcell{大气压强:\ } & \addcell{\SI{99.84}{kPa}}\\
        \cline{2-2}
    \end{tabular}
    \begin{tabular*}{\textwidth}{c}
    \\
    \hline % 分割线
    \end{tabular*}
\end{center}
% 摘要
\textsf{摘\ \ 要}\ \ 本实验使用静态法测量 \ce{CCl4} 在不同温度下的饱和蒸气压。测得 \ce{CCl4} 在当前大气压沸点为 \( T_b = 348.97 \pm 0.02\,\mathrm{K} \),在标准大气压下沸点为 \( T_b = 349.3 \pm 1.2\,\mathrm{K} \),平均摩尔气化热为 \( \Delta_1^{g} H_m = 32.28 \pm 0.08\,\mathrm{kJ\cdot mol^{-1}} \),摩尔气化熵为 \( \Delta_1^{g} S_m = 92.53 \pm 0.23\,\mathrm{J\cdot K^{-1}\cdot mol^{-1}} \)。使用动态法测量 \ce{H2O} 在不同温度下的饱和蒸气压,测得 \ce{H2O} 的在当前大气压下沸点为 \( T_b = 373.53 \pm 0.03\,\mathrm{K} \),在标准大气压下沸点为 \( T_b = 373 \pm 3\,\mathrm{K} \),摩尔气化热为 \( \Delta_1^{g} H_m = 41.01 \pm 0.08\,\mathrm{kJ\cdot mol^{-1}} \),摩尔气化熵为 \( \Delta_1^{g} S_m = 109.8 \pm 0.2\,\mathrm{J\cdot K^{-1}\cdot mol^{-1}} \)。这些结果验证了 \ce{CCl4} 的摩尔气化熵大致符合褚鲁统规则的预测,而 \ce{H2O} 不符合褚鲁统规则。并探索了动态法测定 \ce{H2O} 沸点时不同读数方式所得到的结果的差别,并提出了改进方案。
% 关键字

\noindent\textsf{关键词}\ \ 物理化学实验;饱和蒸气压;摩尔气化热;Clausius-Clapeyron 方程;褚鲁统规则
\end{titlepage}

