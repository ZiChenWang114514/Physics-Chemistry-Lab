\section{结果与讨论}

\subsection{探索实验:动态法测定 \ce{H2O} 沸点时的读数问题与实验改进}

由于在动态法测定 \ce{H2O} 沸点时,温度-气压计的温度示数难以稳定,会不断跳动增加;为了找出最佳的读数时机与方式,我进行了一系列探索实验:
\begin{enumerate}
    \item 在温度缓慢上升平台期的起始与终点分别取点,进行线性拟合,尝试通过物理量的计算结果准确度来比较不同读数方式的优劣。但遗憾的是,由于动态法存在较大的读数误差,这样的误差传递到拟合直线中,造成了参数 $a$ 不确定度显著大于静态法的参数。而参数 $a$ 又与有明确文献值的摩尔汽化焓 $\Delta_1^{g} H_{m}$ 直接相关。$\Delta_1^{g} H_{1,m}, \Delta_1^{g} H_{2,m},\Delta_1^{g} H_{m}$ 均有过大的误差,导致彼此之间没有显著性差别,无法通过实验结果的准确度来衡量读数方式的优劣,也无法做出最终的判断。
    \item 最终体系连通环境后,测定水在环境气压下的沸点时读数时间的选取。这一部分也面临着同样的不确定度过大的问题,无法得出明确结论。
\end{enumerate}

虽然两项探索实验均没有通过不同读数方式准确性的比较,得出最终孰优孰劣的结论,但是通过进行探索实验,积累了一定的实验经验,基于这些经验,可以提出如下两方面可能的动态法实验改进方法:
\begin{enumerate}
    \item \textbf{控温方式的改进}:在原本的实验\cite{pcl2002}中,并没有明确说明在一定气压下 \ce{H2O} 沸腾的平台期如何控温,为此,我建议:在一定气压下 \ce{H2O} 将要沸腾时,将加热套的加热功率逐渐缓缓调低,以免剧烈沸腾;如果 \ce{H2O} 沸腾较为剧烈,可以继续尽可能降低加热套功率,直至能保持水温和沸腾的最低功率,此时,理想情况下水温可以保持恒定。
    \begin{itemize}
        \item 需要注意的是,由于体系温度逐渐升高,其与环境的热交换功率也在提高,加热套的最低加热功率需要相应的提高。
    \end{itemize}
    \item \textbf{读数方式的改进}:在原本的实验\cite{pcl2002}中,只读取1组数据;而通过我提出的控温方式,可以实现在当 \ce{H2O} 温度稳定时,读取3次温度-气压取平均,即为该气压下水的沸点,以显著降低结果的误差,且有望使得动态法实现类似于静态法的精度。
\end{enumerate}

\subsection{结论}

本实验使用静态法测量 \ce{CCl4} 在不同温度下的饱和蒸气压,测得 \ce{CCl4} 的常压沸点为 \( T_b = 348.97 \pm 0.02\,\mathrm{K} \);根据实验数据作出 \( p-T \) 曲线和 \( \ln \left( \dfrac{p}{p^\ominus} \right) \) 拟合直线,根据拟合直线计算 \ce{CCl4} 的标准大气压下的沸点为 \( T_b = 349.3 \pm 1.2\,\mathrm{K} \),平均摩尔气化热为 \( \Delta_1^{g} H_m = 32.28 \pm 0.08\,\mathrm{kJ\cdot mol^{-1}} \),摩尔气化熵为 \( \Delta_1^{g} S_m = 92.53 \pm 0.23\,\mathrm{J\cdot K^{-1}\cdot mol^{-1}} \)。各热力学数据计算值与文献参考值较为接近。

使用动态法测量 \ce{H2O} 在不同温度下的饱和蒸气压,测得 \ce{H2O} 的在当前大气压 \( 99.84\,\mathrm{kPa} \) 下的沸点为 \( T_b = 373.53 \pm 0.03\,\mathrm{K} \);根据实验数据作出 \( p-T \) 曲线和 \( \ln \left( \dfrac{p}{p^\ominus} \right) \) 拟合直线,根据拟合直线计算 \ce{H2O} 的标准大气压下的沸点为 \( T_b = 373 \pm 3\,\mathrm{K} \),平均摩尔气化热为 \( \Delta_1^{g} H_m = 41.01 \pm 0.08\,\mathrm{kJ\cdot mol^{-1}} \),摩尔气化熵为 \( \Delta_1^{g} S_m = 109.8 \pm 0.2\,\mathrm{J\cdot K^{-1}\cdot mol^{-1}} \)。各热力学数据计算值与文献参考值较为接近。


本实验验证了 \ce{CCl4} 的摩尔气化熵大致符合褚鲁统规则的预测,而 \ce{H2O} 不符合褚鲁统规则,分析可能原因是 \ce{H2O} 中分子间氢键的存在使得摩尔气化熵偏离褚鲁统规则。并探索了动态法测定 \ce{H2O} 沸点时不同读数方式所得到的结果的差别,从控温、读数两方面提出了改进动态法测沸点实验,提升准确度与精密性的改进方法。


